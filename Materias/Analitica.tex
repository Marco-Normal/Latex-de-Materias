\chapter{Química Analítica}
\section{Mediadas de Concentração}
Claramente se faz muito importante sabermos calcular as concentrações de soluções. Precisamos então
de uma medida padrão para conseguirmos padronizar isso. Uma das mais uteis é a normalidade de forma 
\begin{equation}
    N=\frac{m}{EqV}
\end{equation}
onde o \(Eq\) é o equivalente grama, de formula 
\begin{equation}
    Eq=\frac{MM}{x}
\end{equation}
Onde \(MM\) é a massa molar e x pode ser calculado por uma regra de mão
\begin{itemize}
    \item ácidos: x= numero de hidrogênios ionizáveis
    \item Bases: x = numero de hidroxilas
    \item Sais: x= numero de cargas positivas ou negativas
    \item Elementos químicos: x=Valência
    \item Oxidantes e redutores: x= Variação do nox
\end{itemize}
Ex: Qual a massa necessária e \(H_2SO_4\) \(d=1,836 gmL^{-1}\) 98 \% para prepararmos uma sol de 250
mL de uma solução 0,1N de ácido. \(MM=98g Mol^ {-1}\)
\begin{align}
    Eq&=\frac{MM}{x}, x=2\\
    &=\frac{98}{2}\\
    N&=\frac{m}{EqV}\\
    0.1&=\frac{m}{49*0,250}\\
    m&=1.225 g\\
    v&=\frac{1.225}{1,836*0.98}\\
    &=0.68 mL
\end{align}
A unidade usada no S.I é a molaridade, que é o numero de mols por volume da sol, ou seja
\begin{equation}
    M=\frac{m}{MM V}
\end{equation}
 Partes por milhão (ppm), por sua vez, é muito utilizada quanto trabalhamos com concentrações muito
 baixas, ou seja, dentro da nossa solução existe uma quantidade quase ínfima de determinado
 composto. Um exemplo, é medições na atmosfera, por exemplo, onde vamos ter, por exemplo, partes por
 milhão de chumbo. Similarmente temos a parte por bilhão (ppb).

 \subsection{Equilíbrio Iônico}
 Se faz de muita impotência a analise de equilíbrio químico e seu funcionamento. Na pratica de
 engenharia química, a analise de equilíbrio, em principal na síntese de algum produto ou em
 reatores, pode ser de fundamental importância para otimizações e corte de custos além de ser a
 diferença quanto a possibilidade ou nao de uma reação acontecer. \par

 Nosso foco maior sera em equilíbrio iônico, em especifico reações reversíveis. O equilibro químico
 é característico de reações reversíveis, ou seja:
 \begin{align}
 \ch{
 aA + bB <=> [v1][v2] cC + dD}
 \end{align}
 se \(v_1=v_2\), temos
 \begin{align}
     \ch{k1 [A]^a [B]^b <=> k2 [C]^c [D]^d}
 \end{align}
 No equilíbrio:
 \begin{equation}
     \frac{[C]^c[D]^d}{[A]^a[B]b}=\frac{k_1}{k_2}=\mathbf{k}
 \end{equation}
 Pelo principio de Le Chatelier, se uma reação for perturbada por algum fator externo, ela tende a
 espontaneamente se mover para um equilíbrio que tende a anular essa pertubação, procurando
 restaurar esse equilíbrio. Alguns efeitos que levam a essa pertubação envolvem
 \begin{itemize}
     \item Mudança de pressão do sistema
     \item Mudança de temperatura do sistema
     \item Adição ou remoção de produtos ou reagente do sistema
 \end{itemize}
 Ex:
 \begin{align}
     \ch{
     PbCl2_{(s)} &<=> Pb_{(aq)}^{2+} +  2 Cl^-_{(aq)}
     }\\
     K&=\frac{\ch{[Pb^{2+}][Cl^-]^2}}{\ch{[PbCl2]}}
 \end{align}
 
 Adicionando \ch{Cl-} o equilíbrio vai para o reagente.
 
 \subsection{Constantes importantes}
 Ionização da água
 \begin{align}
     \ch{
     H2O + H2O &<=>H3O+ + OH- 
     }\\
     K&=\frac{\ch{[H3O^+][OH^-]}}{\ch{[H2O][H2O]}}\\
     Kw&=\ch{[H^+][OH^-]}=1 \cdot 10^{-14}\\
     pH&=-\log \ch{[H^+]}\\
     pOH&= -\log \ch{[OH^- ]}\\
     pOH+pH&=14  
\end{align}
Mas por que se deu um salto onde se perdeu as concentrações da água? O mais correto de se analisar na
equação de k, o ideal seria analisar as atividades das especies químicas, pois imagine que temos uma
concentração muito baixa no denominado, nosso k iria pro infinito, caso o numerador não acompanhe.
Ent~ai, levando em conta a atividade, temos
\begin{equation}
    k=\frac{aA aB}{aC aD}
\end{equation}
onde \(a\) é a atividade da especie,onde ela esta relacionado com o coeficiente de atividade
\(f[i]\). para substancias pura, \(a=1\), para ions \(a=[ion]\) e para gases \(a=\text{Pressão
parcial do gas}\). Voltando a questão do pH, so faz sentido pensar numa escala de 0 a 14. Não esta
errado o pH ser errado, mas na nossa escala não faz sentido usa-la. Quando vamos calcular o pH temos
de tomar um certo cuidado, pois para concentrações muito pequenas de ácido, vindo de outros meios,
como ácidos, se a concentração de ácido for muito pequeno, teremos de considerar o \ch{[H^+]} da
água. Para uma concentração de \ch{HCl} de \(5 \cdot 10^-8\), usando puramente o log, nosso pH daria
cerca de 7.3. Mas independente da concentração, adicionar ácido sempre tem que diminuir o pH.
Quando consideramos o \ch{[H^+]} da água, que é\(1 \times 10^-7\), teremos um pH de 6.8, que é o esperado,
pois reduziu-se o pH
\subsection{Constante para ácido fraco}
Considerando um ácido fraco, como o acético, vamos ter
\begin{align}
    \ch{
    HAc_{(aq)}&<=>H+ _{(aq)} + Ac- _{(aq)}
    }\\
    K_a&=\frac{\ch{[H^+][Ac^-]}}{\ch{[HAc]}}\\
    K_a &\approx \frac{\ch{[H^+]^2}}{\ch{[HAc]}}\\
    \ch{[HAc]_0} &\approx \ch{[HAc]_1}
\end{align}
Onde o termo final diz que a concentração antes e depois do equilíbrio sao próximas. Isso é devido
que ele nao consegue se dissociar completamente, em fato ele se ioniza muito pouco, então pode ser
feita essa aproximação sem um erro muito grande.

\subsection{Hidrolise de Sais}
Considerando a formação de um sal feito por uma recão de uma base forte e um ácido fraco. Como
exemplo sera o acetato de sódio
\begin{align}
    \ch{NaAc_{(s)} &<=> Na+ _{(aq)} + Ac- _{(aq)}}\\
    \ch{Ac-_{(aq)} + H2O &<=> HAc + OH-}\\
    K&=\frac{\ch{[HAc][OH^-]}}{\ch{[Ac^-][H2O]}}, a_{\ch{H2O}}=1\\
    Kh&=\frac{\ch{[HAc][OH^-]}}{\ch{[Ac^-]}}\\
    Kh &\approx \frac{\ch{[OH^-]^2}}{\ch{[Ac^-]}}\\
    Kh&=\frac{Kw}{Ka}
\end{align}
\(Ka\) é a constante do ácido que esta sendo formado. Se mantém a consideração sobre \ch{[Ac]^-},
que antes e depois do equilíbrio sao aproximadamente iguais.
\subsection{Solução Tampão}
Inicialmente vamos supor que voce precise fazer uma reação tal que ela é extremamente sensível a
mudanças de pH. Voce ja fez um teste e viu que ao longo da reação ela varia de pH. Então é
necessário que se mantenha o pH estável. \par

Para isso se tem a solução tampão, tal que sua ideia principal é que com a adição de ácidos ou
bases, se tenha uma variação nula ou quase zero do pH da solução. Um jeito fácil de preparar é
colocando o par ácido ou base fraca e seu ion. Isso acontece por causa do efeito do ion comum, onde
tendo uma especie comum se manipula o equilíbrio tendendo a manter o pH. Ex:
\begin{align}
    \ch{
    NaAc_{(s)} &<=> Na- _{(aq)} + Ac- _{(aq)}\\
    HAc _{(aq)} &<=> H+ _{(aq)} + Ac- _{(aq)}
    }\\
    Ka&=\frac{\ch{[ H+ ][ Ac- ]}}{\ch{[HAc]}}\\
    \ch{[ H+ ]}&=\frac{\ch{Ka[HAc]}}{\ch{[ Ac- ]}}\\
    \intertext{Aplicando o log e nomeando, temos}\notag \\
    pH&=pKa + \log \frac{\text{[Sal]}}{\text{[ácido]}}
\end{align}
Vamos ver o que acontece se colocarmos um ácido nela
\begin{align}
    \ch{
    NaAc_{(s)} &<=> Na- _{(aq)} + Ac- _{(aq)}\\
    HAc_{(aq)} &<=> H+ _{(aq)} + Ac- _{(aq)}\\
    HCl_{(aq)}&<=> H+ _{(aq)} + Cl- _{(aq)}
    }\\
\end{align}
Temos uma fonte introduzindo \ch{H+} que é o ácido e uma fonte retirando que é a formação do ácido
fraco então o pH permanece aproximadamente constante. Vamos ver adicionando uma base forte
\begin{align}
    \ch{
    NaAc_{(s)} &<=> Na- _{(aq)} + Ac- _{(aq)}\\
    HAc_{(aq)} &<=> H+ _{(aq)} + Ac- _{(aq)}\\
    NaOH &<=> Na+ + OH-\\
    H2O &<=> H+ + OH-
    }
\end{align}
Havendo uma dissociação do ácido acético para se manter o pH do meio.
 