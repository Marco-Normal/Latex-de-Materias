\chapter{Termodinâmica}
\section{TDQ 2}
\subsection{Termodinâmica de Soluções}
Inicialmente vamos fazer analise para o caso mais simples que é o de gases ideal. Se faz de extrema
importância a analise do caso ideal pelo fato de servir de base para eventualmente levarmos para o
caso real. Quase todas as equações e essas analises vem da equação fundamental \(P\Bar{V}=RT\).
Quando fazemos a analise real geralmente colocamos como \(Z=\frac{P\Bar{V}}{RT}=1\) Para o caso
real, claro, esse valor vale 1. Para qualquer outro caso o valor varia. Esse comportamento é ideal
em casos de pressão baixa, onde elas estão muito afastadas e pode-se desconsiderar forcas
intra-moleculares. Para casos \(Z<1\) as pressões são altas, onde tem de se considerar as forças
intra-moleculares. Vamos analisar à equação de van der walls, de formula:
\begin{equation}\label{eq:eq de van der walls}
    P=\frac{RT}{V-b}-\frac{a}{V^2}
\end{equation}
Onde \(b\) é o co-volume que é quase como um fator corretivo, de mesma unidade de \(V\), onde ele
caracteriza o fator de repulsão. \(a\) é o fator das forças de intermoleculares, onde serve para a
correção de pressão, junto com o termo \(\frac{1}{V^2}\). \(a,b\) são especificas para cada
substancia. Rearranjando \eqref{eq:eq de van der walls}, multiplicando-a por \(\frac{V}{T}\), temos:
\begin{equation}
    Z=\frac{V}{V-b}-\frac{a}{RTV}
\end{equation}
Para substancias puras, teremos um ponto de inflexão na isotérmica critica,ou seja
\begin{equation}
    (\frac{\partial p}{\partial V})_T=(\frac{\partial^2}{\partial V^2})_T=0
\end{equation}
Tal relação pode ser utilizada para fazer uma equação em função das propriedades do ponto critico.
\par

Estados correspondentes: Essa teoria corresponde que dois estados terão \(Z\) iguais se possuírem
\(T_r,V_r,P_r\) iguais, onde:
\begin{align}
    \phi&=\frac{V}{V_C}=V_r\\
    \pi&=\frac{P}{P_c}=P_r\\
    \theta&=\frac{T}{T_c}=T_r
\end{align}
A equação de estados reduzidos é
\begin{equation}\label{eq:eq dos gases correspondentes}
    \pi=\frac{8 \theta}{3 \phi -1} - \frac{3}{\phi^2} 
\end{equation}
\subsection{Casos do Diagrama Generalizado dos Gases}
\begin{itemize}
    \item Caso 1: tendo-se \(\pi, \theta \), podemos calcular \(z\) diretamente sendo o caso mais simples de
    ser analisado
    \item Caso 2: Quando não temos \(\pi\), vamos ter que criar uma função \(z=z(\pi)\), tendo-se T
    \begin{align}
        Z&=\frac{P_cV}{nRT}\pi\\
        &=k\pi
    \end{align}
    Onde é uma reta e \(k\) é uma constante. Atribuindo dois valores à pressão, e traçando uma reta.
    Calculando \(\theta\), conseguimos então saber onde ela intercepta na curva desejada.
    Se \ \(\theta=1.1\), olhamos onde a reta corta essa curva em especifico.

    \item  Caso 3: Quando não temos \(T\), vamos precisar criar uma equação \(z=z(\theta)\), onde teremos
    \begin{align}
        z&=\frac{PV}{nRT_c \theta}\\
        &=\frac{h^\prime}{\theta}
    \end{align}
    Escolhendo um mínimo de 3 pontos, conseguimos fazer nossa função. Na interseção com o valor
    conhecido da reta, conseguimos descobrir nosso valor desejado. Essa curva vai ter 2 pontos,a direita
    e esquerda, então ficar esperto. Novamente, com o gráfico, calculamos \(\pi\) e vemos onde o gráfico
    vai bater na curva do nosso \(\pi\) especifico. Se tivermos \(\pi=1.5\) vamos ver onde nosso gráfico
    \(z=\frac{h^\prime}{\theta}\) intercepta a curva de \(\pi=1.5\)
\end{itemize}
\subsection{Misturas}
Quando temos uma mistura de \(n\) substancias, para calcular a \(P_c^\prime,T_c^\prime\) é dado por
\begin{align}
    P_c^\prime&=\sum_i x_i P_{ci}\\
    T_c^ \prime&= \sum_i x_i P_{ci}
\end{align}
Onde \(x_i\) é a fracão molar da substancia na mistura e \(P_{ci}, T_{ci}\) são as pressões e
temperaturas criticas de cada substancia.
\subsection{Relacoes fundamentais entre propriedades}
`E importatnte relembrar que algumas propriedades do gas sao dependentes ou independentes do caminho
ou seja, de forma que independente do caminho, o valor sera o mesmo ou nao. Funcoes dependentes de
caminho sao o calor e trabalho que dependendem da rota que levamos o gas, mas entalpia, entropia e energia 
interna sao independentes do caminho. Retomando a primeira lei da termodinamica podemos escreve-la como \(\mathrm{d}U=\mathrm{d}q \mathrm{w}\),
onde \(\mathrm{d}w_r{rev}=p\mathrm{d}V\) e \(\mathrm{d}{rev}=T\mathrm{d}S\), portanto
\(\mathrm{d}U=T\mathrm{d}S-p\mathrm{d}V\), onde para um trabalho reversivel `e maior que o irreversivel, pois nao se tem perda.
Pela desigualdade de Clausis, para irreversiveis, isso fica claro, ela fala que \(T\mathrm{d}s > \mathrm{d}q\).
Por consequencia de \(U\) ser funcao de \(S,V\), podemos escreve-la em funcao de derivadas parcials
\begin{equation}
    \mathrm{d}U=(\frac{\partial U}{\partial S})v\mathrm{d}S+(\frac{\partial U}{\partial V})_s \mathrm{d}V
\end{equation}
Se a composicao do sistema  for a mesma, vemos que
\begin{align}
    (\frac{\partial U}{\partial S})_V&=T\\
    (\frac{\partial U}{\partial V})_s&=-p
\end{align}
Como sabemos que a forma diferencial da primeira lei da termodinamica `e uma equacao exata, temos que
\begin{equation}
    (\frac{\partial T}{\partial V})_s=-(\frac{\partial P}{\partial s})_v
\end{equation}
Onde temos diversas outras equacoes que relacionam diferentes quantidades, de deducao semelhante.
\subsection{Potencial Quimico}
O potencial quimico \(\mu\) `e chamado tambem de tendencia de escape, tal que em condicoes de se um potencial
quimico for maior que o outro, o de maior tem tendencia a sair do tanque. O \(\mu\) de uma substancia pura se define como
\begin{equation}
    \mu =(\frac{\partial G}{\partial n})_{T,p}
\end{equation}
onde podemos tambem exprimir o potencial quimico de uma reacao como
\begin{equation}
    \mu =\mu _0 +RT \ln (\frac{p}{p_0})
\end{equation}
Onde \(p\) `e a pressao de um gas, \(p_0\) `e a pressao em que \(\mu _0\) `e medido. \(\mu _0\) `e o potencial padrao da substancia . Se tivermos uma mistura, teremos que fazer a media ponderada do \(\mu\) para a mistura.
\subsection{Fugacidade}
A fugaciade \(f\) `e definida como a tendencia de escape de uma determinada fase, ou seja, o quanto ele quer
sair de um determinado estado. Ele tem as mesmas dimensoes da pressao.
\begin{equation}\label{eq:tdq2 mi com fugacidade}
    mu =\mu _0 +RT \ln (\frac{f}{p_0})
\end{equation}
Vamos definir a fugacidade como
\begin{equation} \label{eq:tdq2 fugacidade}
    f=\phi p
\end{equation}
Substituindo-a em \eqref{eq:tdq2 mi com fugacidade}, temos
\begin{equation}
    mu =\mu _0 +RT \ln (\frac{p}{p_0}) + RT \ln \phi
\end{equation}
Lembrando que \(\phi\) `e adimensional. O valor de \(\ln \phi\) pode ser calculado como
\begin{equation}
    \ln \phi = \int_0^p (\frac{Z-1}{p}) \ \mathrm{d}p
\end{equation}
Onde \(Z\) `e o nosso fator de compressibilidade do nosso gas. \par
Estime a fugacidade do nitrogenio \(500 \ atm, 0 ^\circ \). \(P_c=33,5 \ atm, 126 \ k\)  
\begin{align}
    \theta &= \frac{0+273}{126,2}\\
    &=2.1632\\
    \pi &= \frac{500}{33,5}\\ 
    &= 14.9253
    \intertext{Por estimacao no grafico,\(\phi \approx 1.24\)}\notag \\
    &\therefore\\
    f&= \phi \cdot P\\
    &=1.24 \cdot 500\\
    &=620 \ \text{atm}
\end{align}
\subsection{A solucao Ideal}
A solucao ideal pode ser definida como aquela solucao que segue o comportamento linear na relacao pressao x fracao molar.
Ou seja, \(p =p^\circ(1-x_2), x + x_2 =1\) onde \(x\) é a fracao molar do solvente e \(x_2\) e a fracao molar do soluto. Isso vem de que a adicao
de um soluto nao volatil a um nao volatil faz com que se altere a pressao da solucao. A adicao de um soluto nao volatil faz com que
se diminua a pressao de vapor da solucao. Pela Lei de Raoult \(p = xp^{\circ}\) e substituindo que \(x +x_2 =1\), temos nossa relacao 
anteriormente dita. Ademais \(p^\circ\) é a pressao de vapor do liquido puro. Para uma solucao ideal com \(n\) componentes a pressao
total se da por 
\begin{equation}
    P = \sum^n p_n
\end{equation}
Onde nossa Lei de Raoult se da
\begin{equation}
    p_n = x_n P^\circ
\end{equation}
\subsection{Modelos para a energia de Gibbs}
A ideia de fugacidade se encaixa muito bem pois para o criteiro de equilibrio mesmo é tao simples quanto ao de potencial.
Para obter a relacao com a fugacidade basta igualar os potenciais quimicos das fases \(\alpha, \beta\)
\begin{equation}
    \mu_i^\alpha = \mu_i^\beta
\end{equation}
Nos lembrando de
\begin{align}
    \mu_1 - \mu_0 &= Rt \ln \left[\frac{f_1}{f_1^\alpha}\right]\\
    \mu_1^{\alpha,0} +RT \ln \left[ \frac{f_f^{\alpha}}{f_i^{\alpha}} \right] &= \mu_i^{\beta,0} + RT \ln \left[ \frac{f_f^\beta}{f_i^{\beta}}\right]\\
    \mu_1^{\alpha,0} - \mu_1^{\beta,0} &= RT \ln \left[ \frac{f_f^{\alpha,0}}{f_f^{\beta,0}} \right] + RT \ln \left[ \frac{f_i^{\beta,0}}{f_i^{\alpha,0}}\right]\\
\end{align}
Como definimos a primeira parte da equacao como 0, temos que
\begin{align}
    0 &= RT \ln \left[ \frac{f_i^\beta}{f_i^\alpha} \right]
    &\therefore
    f_i^\alpha &= f_i^\beta
\end{align}
Portanto, podemos definir que 
\begin{align}
    T^\alpha &= T^\beta\\
    P^\alpha &= P^\beta\\
    f_i^\alpha &= f_i^\beta
\end{align}
Para trabalharmos com isso, podemos  preferencialmente escolher a fase gasosa com pressao baixa,
fazendo com que a fugacidade seja igual a pressao. Assim, podemos definir que
\begin{align}
    f_i \to P\\
    \phi_i \to 1
\end{align}
Podemos escrever a expresao da fugacidade para uma substancia pura como
\begin{align}
    g_i \to g_i^\circ = RT \ln \left[\frac{f_i^\circ}{P_{baixa}} \right]\\
    P^\circ = P_{baixa}\\
    T^\circ = T_{sis}
\end{align}
E tambem
\begin{equation}
    \varphi = (f_i^\circ / P_{sis})
\end{equation}
\subsubsection{Propriedades de mistura de uma gas ideal}
O gas ideial fornece o estado de referencia para a fase vapor. Vamos relembrar as propriedades de mistura
de um gas ideal. Desejamos determinar \(\Delta \upsilon_{mist}, \Delta h_{mist}, \Delta s_{mist}, 
\Delta g_{mist} \). Para uma mistura vamos ter que
\begin{align}
    \Delta \upsilon_{mist}^{ideal} &= 0\\
    \Delta h_{mist}^{ideal} &= 0\\
    \Delta s_{mist}^{ideal} &= -R \sum y_i \ln y_i\\
    \Delta g_{mist}^{ideal} &= RT \sum y_i \ln y_i
\end{align}
A variacao da energia livre de Gibbs se da como
\begin{equation}
    \Delta G_{mist} = nrT \sum_i x_i \ln x_i
\end{equation}
Por fim, a entropia para misturas se da como 
\begin{equation}
    \Delta S_{mist} = -nr \sum_i x_i \ln x_i
\end{equation}
E \(\Delta H_{mist} = 0\)
\subsection{Equilíbrio Liquido-Vapor}
Vamos inicialmente colocar a Regra de Duhem, enunciado como
\begin{equation}
    F = C - P + 2
\end{equation}
Onde \(F\) é o numero de graus de liberdade, \(C\) é o numero de componentes e \(P\) é o numero de
fases. Vamos ter que \(F\) sao as quantidades de variáveis intensivas que podem ser alteradas sem
alterar o número de fases em equilíbrio. \par

O teorema de Duhem, por sua vez, enuncia que, se em um sistema com um componente, como a agua pura,
fixamos \(C = 1\), então nossa regra de fases se reduz a \(F = 3- P\). Se apenas uma fase estiver
presente, \(F = 2\), implicando que podemos alterar \(p, T\). Se no sistema tivermos duas fases,
\(F = 1\), implicando que podemos alterar \(p\) ou \(T\), apenas 1 por ver. Com tres fases, \(F =
0\), implicando que nao podemos alterar \(p\) ou \(T\), virando um ponto no nosso diagrama. \par

\subsection{Equilíbrio Liquido-Vapor}
Definimos o ponto de bolha como quando nosso liquido esta na iminência de se vaporizar ou seja,
quando o primeiro traco de vapor se forma. O ponto de orvalho, por sua vez, é quando o primeiro
traco de liquido se forma. \par

Se tivermos um sistema com duas fases, liquido e vapor, em equilibrio, podemos fazer o grafico do
equilibrio do diagrama em funcao \(P, x, y\) ou  \(T, x, y\), onde \(x, y\) sao as fracoes molares
da fase liquida e gasosa respectivamente.\par
%Colocar imagem depois
\subsection{Misturas Azeotropicas}
Existem diversos sistemas binarios que nao seguem a lei de Raoult, apresentando diversos disvios de
forma que nao podemos descrever o sistema como uma mistura ideal. Para sistemas reais, as curvas de
ponto de ebulicao apresentam maximos ou minimos, que correspondem ao maximo e minimo de pressao de
vapor. \par

Quando temos um sistema com um maximo de pressao de vapor, chamamos de azeotropo positivo, e quando
temos um minimo, chamamos de azeotropo negativo. Para os desvios poitivos a lei de Raoult nosso
azeotropo te ponto de ebulicao minimo, ja para desvios negativos, nosso azeotropo tem ponto de
ebulicao maximo.\par

Para sistemas com ponto de ebulicao maior que o das substancias que as formam, temos o azeotrodo de
maximo. Porem, se nosso azeotropo tiver ponto de ebulicao menor que o das substancias que o formam,
ele é chamado de azeotropo de minimo.\par 

Numa coluna de destilacao, se tivermos um azeotropo de maximo, ele sobrara como residuo, enquanto se
tivermos um azeotropo de minimo, ele sera destilado. Caso queiramos saber a composicao a uma
determinada temperatura ou pressao, vamos ignorar um dos lados e nos importar apenas com o outro.
\par

\subsection{Destilacao em Recipientes Abertos}
Para a destilacao com retirada de vapor, sera considerado que todo vapor recolhido tera composicao
final media do primiero traco de vapor, PTV, e do vapor final, durante o intervalo de temperaturas
inicial e final, logo:
\begin{equation}
    y_{medio} = \frac{y_{i} + y_{final}}{2}
\end{equation}
Onde \(y_{i}\) é a composicao do primeiro traco de vapor e \(y_{final}\) é a composicao do vapor
na temperatura final. \par
%colocar exemplos e imagens aqui depois
\subsection{Equilibrio de Fases}
Inicialmente é valido comentar sobre a condicao de equilibro, em que é necessario que \emph{O
potencial quimico de cada constituinte deve possuir o mesmo valor em todas as fases em equilibrio}.
Se uma fase tiver mais de um constituinte, o potencial quimico de cada constituinte deve possuir o
mesmo valor em todas as fases em equilibrio. Se um constituente estiver presente em mais de uma
fase, o potencial quimico desse constituinte deve possuir o mesmo valor em todas as fases. \par

Por exemplo, se tivermos um sistema agua e seu vapor, em equilibrio vamos ter que:
\begin{equation}
    \mu_{agua}^{liquido} = \mu_{agua}^{vapor}\\
\end{equation}
Se o sistema for constituido apenas de um componente, temos que 
\begin{align}
    \mu &= \frac{G}{n}\\
    \mathrm{d}G &= - S\mathrm{d}T + V\mathrm{d}P \cdot \left( \frac{1}{n} \right)\\
    \mathrm{d}\mu  &= - \overline{S} \mathrm{d}T + \overline{V} \mathrm{d}P\\
    \begin{dcases}
    -\overline{S} , &\quad \left( \frac{\mathrm{d}\mu }{\mathrm{d}T}  \right)_{p}   ;\\
    \overline{V} , &\quad\left( \frac{\mathrm{d}\mu }{\mathrm{d}p}  \right) _{T} ;\\
    \end{dcases}
\end{align}
\subsection{Estabilidade das fases formadas por uma substancia pura-Influencia da Temperatura}
De acordo com a terceira lei da termodinamica, a entropia de uma substancia é sempre positiva, logo
\begin{equation}
    \left( \frac{\mathrm{d}\mu }{\mathrm{d}T}  \right) _{p} = -\overline{S} < 0
\end{equation}
O grafico de \(\mu \) em funcao de \(T\) é uma reta decrescente, logo, a inclinacao da reta é sempre
negativa. Para os diferentes estados da substancia, vamos ter que
\begin{align}
    \left(\frac{\partial \mu_{solido} }{\partial {d}T}\right)_{p} &= -\overline{S}_{solido} \\
    \left( \frac{\partial \mu_{liquido}}{\partial T}  \right)_{p} &= -\overline{S}_{liquido} \\ 
    \left( \frac{\partial \mu_{gas} }{\partial T}  \right)_{p} &= -\overline{S}_{gas} \\
\end{align}
Onde
\begin{equation}
    \overline{S}_{gas} \gg \overline{S}_{liquido} > \overline{S}_{solido}
\end{equation} 
%Colocar grafico

\subsection{Estabilidade das fases formadas por uma substancia pura-Influencia da Pressao}
Para a influencia da pressao, olhamos a nossa equacao de \(\overline{V} \) ja enunciada, podendo ser
reescrita como
\begin{equation}
    \mathrm{d} \mu = \overline{V} \mathrm{d}P
\end{equation}
Se a pressao diminui, \(\mathrm{d}p \) é negativo, \(\overline{V} \) é positivo, logo,
\(\mathrm{d}\mu \) diminui. Logo, a inclinacao da reta é negativa. \par
%Colocar grafico
Concluimos que quando variamos a pressao, altera-se a temperatura de fusao e ebulicao de uma das
substantancias. Ademais, as temperaturas de ebulicao é muito maior que a de fusao. Se a pressao for
reduzido a um valor suficiententemente baixo de forma que a temperatura de ebulicao seja menor que a
de fusao, o liquido nao tera estabilidade e ocorrera sublimacao do solido. \par

A pressao a qual ocorre a sublimacao pode ser determinada pela equacao chamada Regra de Trouton:
\begin{equation}
    \ln p = -10.8 \left( \frac{T_{eb} -T_{f}}{T_{f}} \right)
\end{equation}
\subsection{Equacao de Clapeyron} 


