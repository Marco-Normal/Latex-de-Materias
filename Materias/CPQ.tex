\chapter{Controle de Processos Químicos}
\label{chap:cpq}
\section{Introdução}
\label{sec:cpq-intro}
Esse professor é maluco da cabeça, ele não segue nenhuma linha de raciocínio então o que será
escrito aqui é apenas uma tentativa de organizar as ideias dele. \par

Imaginemos que tenhamos uma planta com uma entrada mássica com valor \(w\), de um fluido qualquer.
Ele possui uma temperatura \(T_i(t)\) de entrada. Dentro do tanque de armazenamento, existe um
aquecedor, que aquece o fluido até uma temperatura \(T\), qualquer. Se estivermos um regime
estacionário, nossa vazão mássica também sera \(w\). Queremos que nosso regime seja estacionário,
logo, que a temperatura de estado estacionário, \(T_{is}\), que essa será a mais importante de
entendermos. Se formos colocar em variável desvio, teremos que \(T_i(t) = T_i(s) - T_{is}\). \par

Na nossa saída do tanque, nosso líquido terá uma temperatura \(T_{s} = T_R\), ou em variável desvio
\(T^{\prime} (t) = T(t) - T_R\). \footnote{O linha (\(^{\prime} \) ), geralmente utilizada para derivada, nesse caso é
para significar variável desvio, porém, nem sempre isso será verdade.}

%Honestamente é preferível a morte

Se tivermos uma introdução de uma pertubação do estilo degrau, teremos que \(T_i(t) = T_i(s) -
T_{is} + \Delta T_i\), onde
\begin{equation}
    \Delta T_i = \begin{cases}
        T_{is} , & t < 0 \\
        T_{is} + \Delta T_i, & t \geq 0
    \end{cases}
\end{equation}
Vamos querer sempre fazer com que essa variação seja a mais abrupta possível, para que se tenha nos
aproximar o máximo possível de um degrau. Mas claro, que vamos ter um tempo de resposta, que terá um
erro associado, ou seja, para uma dada equação, temos
\begin{align}
    \dot{q}(t) &= \dot{q}_s + k_{c} e(t)\\
    \dot{q}(t) - \dot{q}_s = k_{c} e(t) &= \dot{Q}(t)\\
    k_{c} &> 0\\
    e(t) &= T_R - T(t)\\
\end{align}
Para essa equação em particular, estamos tratando de energia, calor. Se formos fazer um balanço de
energia, temos
\begin{align}
    \dot{q}(t) - w_{c} \left( T(t) - T_{i} (t) \right) = \rho v_{c} \frac{\mathrm{d}T(t)}{\mathrm{d}t}  
\end{align}
Onde em facilmente é apenas a taxa que entra é a taxa que sai.
\section{Transformada de Laplace}
A transformada de Laplace é uma das mais importantes ferramentas matemáticas para a análise de
sistemas dinâmicos. Ela é uma ferramenta que nos permite transformar uma equação diferencial em uma
equação algébrica. Ela é definida como:
\begin{equation}\label{eq: equacao de laplace}
    \mathcal{L} \left\{ f(t) \right\} = F(s) = \int_{0}^{\infty} f(t) e^{-st} \mathrm{d}t
\end{equation}
Onde \(s\) é um número complexo. A transformada de Laplace é uma transformada linear, ou seja,
\begin{equation}
    \mathcal{L} \left\{ a f(t) + b g(t) \right\} = a F(s) + b G(s)
\end{equation}
A transformada de Laplace é uma transformada de uma função de tempo para uma função de frequência,
geralmente com a variável \(s\). Vamos ver alguns exemplos.
\subsection{Exemplo 1}
Seja o PVI dado por
\begin{align}
    \begin{dcases}
        2 \frac{\mathrm{d}\xi (t)}{\mathrm{d}t} + \xi (t) &= 1;\\
        \xi (0) &= 0\\
    \end{dcases}
\end{align}
Resolva esse PVI utilizando a transformada de Laplace. \par
\textbf{Solução:} Aplicando a transformada de Laplace na equação diferencial, temos
\begin{align}
    \mathscr{L} \left\{ 2 \frac{\mathrm{d}\xi (t)}{\mathrm{d}t} + \xi (t) &= 1 \right\}\\
    2 \mathscr{L} \left\{ \frac{\mathrm{d}\xi (t)}{\mathrm{d}t} \right\} + \mathscr{L} \left\{ \xi (t) \right\} &= \mathscr{L} \left[ 1 \right]\\
    2 s \left[\mathscr{L} \left\{ \xi (t) \right\} - \xi (0)\right] + \mathscr{L} \left\{ \xi (t) \right\} &= \frac{1}{s}\\
    2 s \left[\mathscr{L} \left\{ \xi (t) \right\} - 0\right] + \mathscr{L} \left\{ \xi (t) \right\} &= \frac{1}{s}\\
    \mathscr{L} \left\{ \xi (t) \right\} \left( 2s + 1 \right) &= \frac{1}{s}\\
    \mathscr{L} \left\{ \xi (t) \right\} &= \frac{1}{s \left( 2s + 1 \right)}\\
    \mathscr{L} \left\{ \xi (t) \right\} &= \frac{1}{s} - \frac{2}{2s + 1}
\end{align}
%Falta aplicar a inversa
%────────────────────────────────────────────────────────────────────────────────────────────────────────────────────────────────────────────────────
\subsection{Exemplo 2}
Seja o PVI dado por
\begin{align}
    \begin{dcases}
        \frac{\mathrm{d}^2  T(t)}{\mathrm{d}t^2} + 2 \frac{\mathrm{d} T(t)}{\mathrm{d}t} + T(t) = 2 \\
        T(0) = 3\\
        \frac{\mathrm{d} T(t)}{\mathrm{d}t} \bigg|_{t=0} = 1\\
    \end{dcases}
\end{align}
Resolva esse PVI utilizando a transformada de Laplace. \par
\textbf{Solução:} Aplicando a transformada de Laplace na equação diferencial, temos
\begin{align}
    \mathscr{L} \left\{ \frac{\mathrm{d}^2  T(t)}{\mathrm{d}t^2} + 2 \frac{\mathrm{d} T(t)}{\mathrm{d}t} + T(t) = 2 \right\}\\
    \mathscr{L} \left\{ \frac{\mathrm{d}^2  T(t)}{\mathrm{d}t^2} \right\} + 2 \mathscr{L} \left\{ \frac{\mathrm{d} T(t)}{\mathrm{d}t} \right\} + \mathscr{L} \left\{ T(t) \right\} &= \mathscr{L} \left\{ 2 \right\}\\
    s^2 \mathscr{L} \left\{ T(t) \right\} - s T(0) - \frac{\mathrm{d} T(t)}{\mathrm{d}t} \bigg|_{t=0} + 2 s \mathscr{L} \left\{ T(t) \right\} - T(0) + \mathscr{L} \left\{ T(t) \right\} &= \frac{2}{s}\\
    s^2 \mathscr{L} \left\{ T(t) \right\} - 3 s + 1 + 2 s \mathscr{L} \left\{ T(t) \right\} - 3 + \mathscr{L} \left\{ T(t) \right\} &= \frac{2}{s}\\
    \left( s^2 + 2s + 1 \right) \mathscr{L} \left\{ T(t) \right\} &= \frac{2}{s} + 3 s - 4\\
    \left( s + 1 \right)^2 \mathscr{L} \left\{ T(t) \right\} &= \frac{2}{s} + 3 s - 4\\
    \mathscr{L} \left\{ T(t) \right\} &= \frac{2}{s \left( s + 1 \right)^2} + \frac{3 s - 4}{\left( s + 1 \right)^2}\\
\end{align}
% Faltar aplicar a inversa
%────────────────────────────────────────────────────────────────────────────────────────────────────────────────────────────────────────────────────
\subsection{Exemplo 3}
Seja o PVI dado por

