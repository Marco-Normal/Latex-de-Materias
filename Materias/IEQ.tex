\chapter{IEQ}
\section{Inicio e exemplos}
\subsection{A ideia da instrumentação}
Vamos levar nossas mentes para uma época anterior, antes da mecanização do trabalho podemos imaginar
como era, onde tudo era manual, demorava muito tempo para se fazer as coisas. Eventualmente com o
avanço da tecnologia, não é difícil ver que as coisas foram se tornando mais fáceis, ou seja, a
maquina foi substituindo o trabalho do homem e hoje, mais do que nunca, vemos como a automação vem
fazendo o nosso trabalho.\par

Vamos fazer uma distinção:
\begin{itemize}
    \item Mecanização: substituição dos trabalhos manuais por maquinas, ou seja, bombas, transporte
    de fluidos
    \item Automatização: substituição do trabalho mentais por maquinas, ou seja, utilizar
    computadores, maquinas pneumáticas, mecânicas para fazer nosso trabalho.
\end{itemize}
Claro que naturalmente, todas nossas automatizações e mecanizações vieram como modo de deixar a vida
e o trabalho humano mais fácil. Mas por mais que uma grande parte do nosso trabalho tem sido
automatizado, ainda se é necessário que especialista ou alguém em algum nível consiga entender
tanto a maquina tanto o processo, para que se posso entender caso aconteça algo errado ou se
identifique uma forma de melhoria no processo. \par

Um dos grandes avanços que tivemos foi a utilização de sensores que nos ajudam a monitorar as
principais variáveis do nosso processo, como pressão, temperatura, vazão, etc. Esses sensores tomam
medidas ao longo do tempo e retornam os valores, para que sejam analisados, seja por humanos sejam
por outras maquinas que regulariam nosso processo. Com uma implementação de sensores e outros tipos
de automatizações, podemos integra-los ao nosso sistema e aumentar nossa produção. Ademais, ate
mesmo em sentido de segurança, sensores sao extremamente precisos, ou seja, conseguimos medir
variáveis com um grau de precisão altíssima, fazendo com que caso haja, por exemplo, um aumento de
pressão, seja pequena ou grande, conseguimos fazer com que esse aumento, passado um valor de
threshold, ou um valor de segurança, podemos automaticamente desligar nossa caldeira, ou regular sua
temperatura, etc. \par

Vamos agora nos focar um pouco sobre esses aumentos, ou também comumente chamadas de pertubações.
Imaginemos que num sistema sem ter sido perturbada a um tempo grande, ou seja, esta num estado
estacionário, podemos representar sua variável de controle, temperatura, por exemplo, como uma linha
continua ao longo do tempo. Se decidirmos pertusa-la, aumentando ou diminuindo sua temperatura,
vamos ver uma linha reta, ou levemente curvada sob o período de tempo que estamos perturbando-o. Se
fizermos uma pertubação longa, dobrando sua temperatura, por exemplo, veremos uma longa linha
crescente durante o tempo necessário para esse aquecimento. Mas vamos imaginar que desejamos
aumentar em \(0.5\) nossa temperatura. Dado um meio suficientemente potente, conseguimos subir essa
temperatura em um intervalo \(\Delta t\) muito pequeno. Fazendo-o tender a zero, teremos uma função
degrau, que seria uma pertubação ideal para engenharia. Se tivéssemos \(n\) dessas funções, teríamos
uma variação instantânea para nossa temperatura desejada. Porem, isso nao é possível no mundo real,
logicamente, e nossa função degrau é continua e suave ao longo do intervalo \(t+ \Delta t\). Logo,
tendo \(n\) dessas transformações, cada uma com sua variação \(\Delta t\), nao necessariamente
iguais, teremos nosso tempo para atingirmos a variação desejada. \par

\subsection{Diferenças}
\paragraph{Automacão vs Instrumentacao Industrial}
Automação é o estudo técnico para conseguirmos diminuir um menor uso de mao de obra humana e uma
maior participação das maquinas \par

Instrumentação, por sua vez, é o estudo e aperfeiçoamento de controles de processos industriais,
assim como aumentando a segurança das maquinas e consequentemente das pessoas ao redor
\par

\paragraph{Malha baerta vs Fechada}
Malha aberta é aquela na qual o sistema esta agindo sem nenhuma especie de controle, ou seja, nao
tem nenhuma alteração dentro do sistema. Isso nao significa, necessariamente, que o sistema esta
estacionário, podendo ser dinâmico, apenas nao tendo variações.\par

Malha fechada, por sua vez, esta agindo sobre uma condição de controle, ou seja, a variável
controlada é utilizada para controlar e gerir qualquer outras variáveis.