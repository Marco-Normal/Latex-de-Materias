\chapter{Fenômenos de Transporte II}
\section{Equação de calor}
O escopo dessa disciplina vai ser nas formas de troca e passagem de calor por meio da irradiação,
convecção e condução. Vamos iniciar diretamente com a primeira lei da termodinâmica na forma
diferencial
\begin{equation}\label{eq:Primeira lei termo diferencial}
    \mathrm{d}E=\delta Q-\delta W
\end{equation}
Mas em um sistema a energia pode estar disponível em diversas formas, as mais importantes sendo a
mecânica, que é a soma da potencial mais cinética, e interna.
\begin{align}
    \mathrm{d}E&=E_2-E_1\\
    E&=E_c + E_p + E_i 
\end{align}
Substituindo uma na outra
\begin{equation}
    \mathrm{d}E=E_{c2}  + E_{p2}  + E_{i2}- (E_{c1}  + E_{p1}  + E_{i1} )   
\end{equation}
\[
    \therefore
\]
\begin{align}
    \mathrm{d} E&=(\frac{\delta m \upsilon ^{2}_2 }{2}+ \delta mgz_2 +U_2)-(\frac{\delta m \upsilon ^{2}_1 }{2}+ \delta mgz_1 +U_1)\\
    &=\delta m(\frac{ \upsilon ^{2}_2 }{2}+ gz_2 +U_2)-(\frac{ \upsilon ^{2}_1 }{2}+ gz_1 +U_1)\\
    \label{eq:primeira lei antes sub}
    &=\delta m(\frac{ \upsilon ^{2}_2-\upsilon ^{2}_1 }{2}+ g(z_{2}-z_1)  +(U_2-U_1))
\end{align}
\eqref{eq:primeira lei antes sub} em \eqref{eq:Primeira lei termo diferencial}
\begin{align}\label{eq:primeira lei pt 3}
    \delta Q - \delta W =\delta m(\frac{ \upsilon ^{2}_2-\upsilon ^{2}_1 }{2}+ g(z_{2}-z_1)  +(U_2-U_1))\\
    \intertext{Definindo as operações sinais}\notag 
    \text{sinal}(Q)
    \begin{dcases}
        -1, & \text{se sai da vizinhança e vai para o sistema} ;\\
        1, & \text{se sai do sistema e vai para a vizinhança};\\
    \end{dcases}\notag \\
    \text{sinal}(W)
    \begin{dcases}
        -1, & \text{se sai da vizinhança e vai para o sistema} ;\\
        1, & \text{se sai do sistema e vai para o vizinhança};\\
    \end{dcases}\notag 
\end{align}
O trabalho pode ser de dois tipos \(\delta W_f\) que é o trabalho de fluxo de formula:
\begin{equation}
    \delta W_f=\sum_{i} P_i V_i \text{sinal}(W) 
\end{equation}
E trabalho de eixo, denominado por \(\delta W_s\). Nosso trabalho final fica então na forma:
\begin{equation}\label{eq:trabalhos}
    \delta  W=\delta W_s + \delta W_f
\end{equation}
\eqref{eq:trabalhos} em \eqref{eq:primeira lei pt 3}
\begin{align}
    \delta Q \text{sinal}(Q)-(\delta W_s + \delta W_f)=\delta m(\frac{ \upsilon ^{2}_2-\upsilon ^{2}_1 }{2})+ g(z_{2}-z_1)  +(U_2-U_1)\\
    \delta Q \text{sinal}(Q)=\delta m(\frac{ \upsilon ^{2}_2-\upsilon ^{2}_1 }{2}+ g(z_{2}-z_1)  +(U_2-U_1))+(\delta W_s + \delta W_f)
\end{align}
Dividindo-se tudo por \(\delta m\)
\begin{equation}\label{eq:Primeira lei forma geral}
    \delta^\prime  Q \text{sinal}(Q)=(\frac{ \upsilon ^{2}_2-\upsilon ^{2}_1 }{2})+ g(z_{2}-z_1)  +(U_2-U_1)+(\delta^\prime  W_s + \delta^\prime  W_f)
\end{equation}
Onde o \(\delta ^\prime \) indica 'por quantidade de massa', ou especifico. Para um caso em que
\(W_2\) vai para a vizinhança e \(W_1\) é feito sobre o sistema, com \(\delta^\prime  W_s=0\)  nossa
equação fica:
\begin{align}\label{eq:Primeira lei para quase trocadores}
    \delta^\prime  Q \text{sinal}(Q)&=(\frac{ \upsilon ^{2}_2-\upsilon ^{2}_1 }{2}+ g(z_{2}-z_1)  +(U_2-U_1))+(W_2 - W_1)\\
    &=(\frac{ \upsilon ^{2}_2-\upsilon ^{2}_1 }{2})+ g(z_{2}-z_1)  +(U_2-U_1)+(P_2 V_2  - P_1 V_1)\\
    &=(\frac{ \upsilon ^{2}_2-\upsilon ^{2}_1 }{2})+ g(z_{2}-z_1) +(P_2 V_2 +U_2)  - (P_1 V_1 + U_1)\\
    &=(\frac{ \upsilon ^{2}_2-\upsilon ^{2}_1 }{2})+ g(z_{2}-z_1) +(H_2 - H_1)
\end{align}
Onde H é a entropia do sistema. Para trocadores de calor, onde EM=0, nossa lei colapsa para
\begin{equation}\label{eq:Primeira lei para trocadores de calor}
    \delta^\prime  Q \text{sinal}(Q)=(H_2 - H_1)+\delta^\prime W_f
\end{equation}
Para gases ideal, nossa lei fica:
\begin{equation}\label{eq:Primeira lei para gases ideais}
    \delta^\prime  Q \text{sinal}(Q)=C_p (T_2 - T_1)+\delta^\prime W_f
\end{equation}
Onde \(C_p\) é a constante à pressão constante. Para líquidos incompressíveis:
\begin{equation}\label{eq:Primeira lei para liquidos incompressiveis}
    \delta^\prime  Q \text{sinal}(Q)=C(T_2 - T_1)+\delta^\prime W_f
\end{equation}
Como uma ultima arrumação na nossa equação, multiplicaremos \eqref{eq:Primeira lei para gases
ideais} por \(\delta m\) nos restando com:
\begin{equation}\label{eq:Primeira lei para gases ideais não por massa}
    \delta  Q \text{sinal}(Q)=\delta m C_p (T_2 - T_1)+\delta W_f
\end{equation}
E em trocadores, geralmente \(\delta W_f\) é nulo, logo:
\begin{equation}\label{eq:Primeira lei para gases ideais não por massa trocador}
    \delta  Q \text{sinal}(Q)=\delta m C_p (T_2 - T_1)
\end{equation}

\section{Transferências de calor}
Transferências de calor e definida como a energia em transito por causa de um gradiente de
temperatura que cocore, seja entre 2 meios, seja entre objeto e o meio, etc. A temperatura, por sua
vez, pode ser definida como a medida da agitação das moléculas, ou seja, sua energia cinteira. Na
industria química, isso e mais importante no sentido dos trocadores de calor,logo me desdobarei mais
sobre isso. Tais trocadores de calor podem ser distinguidos em dois, aqueles que tem contato com, os
de contato direto e os sem contato. Os sem contatos são aqueles que não tem contato físico entre o
produto e o meio aquecedor, ou refrigerante, de forma que são separados por alguma membrana ou
parede fina. Os com contado,por sua vez, tem o contato direto entre o produto e o meio
aquecedor/refrigerador. Fica a curiosidade do leitor olhar exemplos sobre cada um.
\subsection{Processo de transferência de calor}

O foco maior, nessa disciplina,vem na parte da energia térmica, em especifico, na energia
calorifica. Essa energia pode ser considerada como a energia interna na sua forma latente e
sensível. Para um processo em que ha troca de calor, sem mudança de fase, o calor pode ser
determinado por:
\begin{equation}
    Q=\delta mc(T_s-T_e)
\end{equation}
onde \(\dot{m}\) e o fluxo mássico, \(c\) e uma constante especifica par cada material e \(T_s,T_e\)
são as temperaturas de Saida e entrada respectivamente. Não so de calculo vive a engenharia,
conhecer como ela é transmitida, e seus mecanismos físicos ainda se mostra vital, tanto para analise
de custo, quanto par projetar, quanto para calcular as taxas de transferências. Como j`a foi dito,
temos 3 grandes formas de transmissão, a condução, convecção e irradiação. Vamos nos destrinchar um
pouco sobre cada uma.
\section{Condução}
O calor nessa forma de transmissão é transmitido necessariamente pelo contato e a nível molecular Ele
é transmitido majoritariamente por meio da transmissão das vibrações das moléculas do material mais
"quente". é importante a ressalva que ela é majoritária em principal em solido opacos. Ela ocorre em
líquidos e gases e em sólidos translúcidos ocorre, mas não é majoritária. Em um exemplo, em um dia
quente, numa casa a condução vai ocorrer através da espessura das paredes. Sob uma analise
matemática
\begin{align}\label{eq:equacao de conducao}
    q_{cond} &\propto A\frac{\mathrm{d}T}{\mathrm{d}x}\\
    q_{cond}&=-KA\frac{\mathrm{d}T}{\mathrm{d}x}
\end{align}
onde\(K\) é uma constante de proporcionalidade relacionada ão material, suas unidades são
\([\frac{W}{mk}]\). \(A [m^2]\) é a area transversal \(T [K]\) é a temperatura\(x [m]\) é a espessura
da parede \(q_{cond}\) é o calor conduzido em \(W\). O menos surge por causa que o calor "flui" do
que tem maior temperatura para o de menor temperatura.
 \par
 Ex: Se temos uma placa de aco-inox de 1 cm de espessura, \(k=\frac{17 W}{mC}\). \(\Delta T=-30 C\)
 Calcule \(q_{cond}/A\):
 \begin{align}
     \frac{q_{cond}}{A}&=\frac{-K \Delta T}{\Delta x}\\
     &=\frac{-17 (30)}{0.01}\\
     &=34 \frac{kW}{m^2}
 \end{align}
 \subsection{Convecção}
 é a transferência de calor dado o movimento macroscópico da substancia indo de uma região quente
 para uma fria. Sua formula pode ser dada como:
 \begin{align}
     q_{conv} &\propto A(T_s-T_{\infty})\\
     q_{conv}& =hA(T_S-T_{\infty})
 \end{align}
Onde \(h= [\frac{W}{m^3K}]\) é o coeficiente de transferência de calor, \(A= [m^3]\) é a area de
contato do fluido com a superfície, \(T\) é a temperatura da superfície, \(T_{\infty}\) é a
temperatura do fluido. \(h\) é determinado empiricamente. Ex: determine \(q_{conv}/A\) para uma
placa com \(T_{\infty}=20,T_s=120,h=10\),
\begin{align}
    q_{conv}/A&=h(T_s-T_{\infty})\\
    &=10(120-20)\\
    &=1\frac{kW}{m^2}
\end{align}
\section{Radiação}
A sua principal diferença entre as duas outras é o fato de ela não precisar de meio físico, ou seja,
ela se propaga por meio do vacuo através de ondas eletromagnéticas. Qualquer corpo sempre esta
irradiando, então se faz sim de absoluta importância o estudo desse processo para a engenharia.
Quando recebemos a radiação, pode-se ter 4 formas de interação.
\begin{itemize}
    \item Absorvida: O meio tem interação com a radiação aumentando sua energia interna \(\rho\)
    \item Transmissão: A radiação atravessa o meio \(\alpha\)
    \item Reflexão: radiação n tem interação mas reflete do meio \(\tau\)
\end{itemize}
Num meio semi-transparente
\begin{equation}
    G=G_r+G_a+G_t
\end{equation}

\begin{equation}
    \rho + \alpha + \tau =1
\end{equation}

Num meio opaco

\begin{equation}\label{eq:irradiacao corpo opaco}
    \rho + \alpha =1
\end{equation}

Com esse informação,podemos definir \(J\) como a radiosidade, que é a taxa que radiação deixa a
superfície, por area
\begin{equation}
    J=E+\rho G
\end{equation}
Onde
\begin{equation}
    E=\varepsilon \sigma T^4_A
\end{equation}
\(\sigma\) é uma constante com valor \(5.669 10^-8 W/(m^2K^4)\) \(\varepsilon\) é a constante de
permissividade do meio. Definindo o fluxo termo radiação que é a taxa liquida de radiação deixando
uma superfície por unidade de area, para um corpo opaco, temos
\begin{align}
    q^{\prime \prime}_{rad}&=J-G\\
    &=E+\rho G-G\\
    &=E-G(1-\rho)\\
    &=E-\alpha G\\
    &=\varepsilon \sigma T^4_s - \alpha G
\end{align}
Conseguimos substituir\(1-\rho\) por \(\alpha\), pois num corpo opaco, que é o caso do nosso estudo,
temos como referencia \eqref{eq:irradiacao corpo opaco} Ex: calcule \(q^{\prime \prime}\) dado
\(A=100 m^2\), \(\varepsilon=0.06\) \(T=37 C\)
\begin{align}
    q^{\prime \prime}&=\sigma \varepsilon A T^4_A\\
    &=5.669 \cdot 10^-8 \cdot 0.06 \cdot 100 \cdot (37+273)^4 \cdot\\
    &=3141.2643 W
\end{align}
\section{Equação geral da condução}
\subsection{Geometria plana}
Essa equação é a geral para todos os sólidos a qual podemos calcular o fluxo calorifico que 
tem por condução.para deriva-la, podemos realizar o balanco de energia para um solido que 
esta apenas sobre efeito de condução. Considerando um elemento de volume com lados 
\(\delta x, \delta y, \delta z\). A analise de energia é dado por 
\begin{equation}\label{eq: balanco de energia}
\begin{split}
    \textit{Taxa de condução dentro do v.c + taxa de geração dentro do v.c.}=\\
    \textit{taxa de condução fora d v.c. + taxa de armazenamento dentro do v.c}
\end{split}
\end{equation}
Vamos representar cada termo dentro de \eqref{eq: balanco de energia}
\begin{equation}\label{eq: calor gerado v.c.}
    \textit{taxa de geração de calor no v.c.}=\dot{q_{GER}}\delta x \delta y \delta z
\end{equation}

\begin{equation}\label{eq: calor armazenado v.c.}
    \textit{taxa de armazenamento no v.c.}=\frac{\partial}{\partial t}(\rho c T) \delta x \delta y \delta z
\end{equation}

\begin{equation}\label{eq: taxa de conducao dentro v.c.}
    \textit{taxa de condução dentro do v.c.}=q^{\prime \prime}_x \delta y \delta z
\end{equation}

\begin{equation}\label{eq: taxa de conducao fora v.c.}
    \textit{taxa de condução para fora do v.c.}= q^{\prime \prime}_{x+\delta x}\delta y \delta z
\end{equation}

\begin{align}
    q^{\prime \prime}_{x+\delta x}=q^{\prime \prime}|_x + \frac{\partial}{\partial}[q^{\prime \prime}|_x \delta x]
\end{align}

 Fazendo \eqref{eq: taxa de conducao dentro v.c.} - \eqref{eq: taxa de conducao fora v.c.}

 \begin{align}\label{eq: diferenca conducao x}
     \eqref{eq: taxa de conducao dentro v.c.}_x - \eqref{eq: taxa de conducao fora v.c.}_{x+ \delta x} &= q^{\prime \prime}_x \delta y \delta z - (q^{\prime \prime}_x + \frac{\partial}{\partial x}[q^{\prime \prime}_x\delta x] \delta  y \delta  z) \\
     &=-\frac{\partial}{\partial x} [q^{\prime \prime}_x\delta x] \delta y \delta z
 \end{align}

 Similarmente para y

 \begin{align}\label{eq: diferenca conducao y}
     \eqref{eq: taxa de conducao dentro v.c.}_y - \eqref{eq: taxa de conducao fora v.c.}_{y+ \delta y} &= q^{\prime \prime}_y \delta x \delta z - (q^{\prime \prime}_y + \frac{\partial}{\partial y}[q^{\prime \prime}_y\delta y] \delta  x \delta  z) \\
     &=-\frac{\partial}{\partial y} [q^{\prime \prime}_y\delta y] \delta x \delta z
 \end{align}

E para z

 \begin{align}
     \eqref{eq: taxa de conducao dentro v.c.}_z - \eqref{eq: taxa de conducao fora v.c.}_{z+ \delta z} &= q^{\prime \prime}_z \delta x \delta y - (q^{\prime \prime}_z + \frac{\partial}{\partial z}[q^{\prime \prime}_z\delta z] \delta  x \delta  y)\\
     &=-\frac{\partial}{\partial z} [q^{\prime \prime}_z\delta z] \delta x \delta y
 \end{align}

 Substituindo \eqref{eq: taxa de conducao fora v.c.},\eqref{eq: taxa de conducao dentro v.c.}, \eqref{eq: calor armazenado v.c.}, \eqref{eq: calor gerado v.c.} em \eqref{eq: balanco de energia} e isolando \eqref{eq: calor armazenado v.c.}

\begin{flalign}
    \begin{split}
        \frac{\partial}{\partial t}(\rho \cdot c \cdot T) \ \delta x \delta y \delta z &= \dot{q}_{GER} \ \delta x \delta y \delta z  \\ 
        &-(\frac{\partial}{\partial x} [q^{\prime \prime}_x\delta x] \delta y \delta z +
        \frac{\partial}{\partial y} [q^{\prime \prime}_y\delta y] \delta x \delta z  +\frac{\partial}{\partial z} [q^{\prime \prime}_z\delta z] \delta x \delta y)
    \end{split} 
\end{flalign}
Dividindo toda equação por \(\delta x \delta y \delta z\)
\begin{equation}\label{eq: fentran2 desenvolvimento conducao}
    \frac{\partial}{\partial t}(\rho \cdot c \cdot T) = \dot{q}_{GER}-(\frac{\partial}{\partial x} [q^{\prime \prime}_x]+ \frac{\partial}{\partial y} [q^{\prime \prime}_y] +\frac{\partial}{\partial z} [q^{\prime \prime}_z])
\end{equation}

 Lembrando da lei de Fourier para condução:
 \begin{equation}
     q^{\prime \prime}_x=-k \frac{\partial T}{\partial x}
 \end{equation}
 Similarmente para as outras coordenadas. Substituindo  em \eqref{eq: fentran2 desenvolvimento conducao}

 \begin{equation}
     \frac{\partial}{\partial t}(\rho \cdot c \cdot T)=-[\frac{\partial}{\partial x}(-k \frac{\partial T}{\partial x})+ \frac{\partial}{\partial y}(-k \frac{\partial T}{\partial y})+ \frac{\partial}{\partial z}(-k \frac{\partial T}{\partial z})] + \dot{q}_{GER}
 \end{equation}
 Dividindo toda equação por k e ajeitando os termos das parciais, vamos ter

 \begin{align}
     \frac{\rho c}{k}\frac{\partial T}{\partial t}&= [\frac{\partial^2 T}{\partial x^2}+ \frac{\partial^2 T}{\partial y^2}+ \frac{\partial^2 T}{\partial z^2}]+\frac{\dot{q}_{GER}}{k} \\
     \label{eq: fentran2 equacao geral de conducao para paredes planas}
     \frac{1}{\alpha}\frac{\partial T}{\partial t}&= [\frac{\partial^2 T}{\partial x^2}+ \frac{\partial^2 T}{\partial y^2}+ \frac{\partial^2 T}{\partial z^2}]+\frac{\dot{q}_{GER}}{k}
 \end{align}

 Para um caso permanente, onde a temperatura nao varia com o tempo, nossa equação se reduz a:

 \begin{equation}\label{eq: fentran2 equacao conducao permanente}
     [\frac{\partial^2 T}{\partial x^2}+ \frac{\partial^2 T}{\partial y^2}+ \frac{\partial^2 T}{\partial z^2}]+\frac{\dot{q}_{GER}}{k}=0
 \end{equation}

 Similarmente, para geometrias infinitas, ou seja, onde podem ser deprecadas outras dimensões,
 como placas, cilindros infinitos e esferas, nossa equação se reduz a:
 \begin{equation}\label{eq:fentran2 eq conducao geometria infinita}
     \frac{\partial^2 T}{\partial x^2}+\frac{\dot{q}_{GER}}{k}=0
 \end{equation}
 Para uma placa sem geração de calor, nossa equação colapsa para
 \begin{equation}
     \frac{\partial^2 T}{\partial x^2}=0
 \end{equation}
Se fossemos fazer uma analise para uma placa plana, ate mesmo pela própria equação, sua temperatura é 
exclusivamente função da sua coordenada, e sua transferência de calor é exclusivamente por essa
coordenada. Logo, conseguimos descobrir sua função e consequentemente sua taxa de transferência.
Partindo de \eqref{eq: fentran2 equacao geral de conducao para paredes planas} e integrando-a duplamente, temos
\begin{align}
    \frac{\partial ^2 T}{\partial x^2}&=0
    \begin{dcases}
    x=0 &, T=T_{s1}\\
    x=L &, T=T_{s2}
    \end{dcases}\\
    \int \frac{\partial ^2 T}{\partial x^2}&=\int 0\\
    \frac{\partial T}{\partial x}&=C_1\\
    \int \frac{\partial T}{\partial x}&=\int C_1\\
    T&=C_1x+C_2
    \intertext{Colocando nossas condicoes de contorno}
    C_2&=T_{s1}\\
    C_1&=\frac{(T_{s2}-T_{s1})}{L}\\
    & \therefore \notag\\
    T&=\frac{x}{L}(T_{s2}-T_{s1}+T_{s1})\\
    \intertext{Colocando na equacao de fourier para calor}\notag\\
    q_x&=-KA\frac{\mathrm{d}T}{\mathrm{d}x}\\
    &=-KA \frac{\mathrm{d}}{\mathrm{d}x}[\frac{x}{L}(T_{s2}-T_{s1})+T_{s1}]\\
    &=-KA\frac{(T_{s2}-T_{s1})}{L}
\end{align}
Podemos então fazer uma relação dessa equação com circuitos elétricos, em que a nossa corrente seria
a taxa de condução, nossa ddp a diferença de temperatura e nossa resistência nossa parede, podemos
escrever então
\begin{align}
    q_x &= \frac{T_{s1} - T_{s2}  }{\left[ \frac{L}{KA} \right] }\\
    &= \frac{T_{s1} T_{s2} }{R_T}
\end{align}
Onde \(R_T\) é
\begin{equation}
    R_T = \frac{L}{KA}
\end{equation}
Portanto, podemos relacionar associação de paredes, tanto em serie, quanto em paralelo, assim como
fazíamos em circuitos elétricos. Se tivermos uma associação em serie, somamos nossas resistências,
ou seja
\begin{align}
    R_T &= \sum_{i} \frac{L_i}{K_i A}\\
    &= \sum_{i} \frac{\Delta x_i}{K_i A}
\end{align}
Assumindo que a area que atravesse seja a mesma e todas estejam na mesma direção do calor. Em
paralelo, por sua vez, vamos ter
\begin{align}
    R_T &= \sum_{i} \frac{1}{R_i}\\
    &= \sum_{i} \frac{K_i A}{\Delta x_i}
\end{align}
\subsection{Equações da Condução para geometria Cilíndrica}
A ideia da demonstração é a mesma, então ressaltando os pontos diferentes. As coordenadas, sendo
cilíndricas, são \(r \delta \phi, \delta r, \delta z\). Portanto, nossas equações ficam
\begin{equation}\label{eq: calor gerado v.c. cilindrica}
    \textit{taxa de geração de calor no v.c.}=\dot{q_{GER}}r \delta \phi \delta r \delta z
\end{equation}
\begin{equation}\label{eq: calor armazenado v.c. cilindrica}
    \textit{taxa de armazenamento no v.c.}=\frac{\partial}{\partial t}(\rho c T) r \delta \phi \delta r \delta z
\end{equation}

\begin{equation}\label{eq: taxa de conducao dentro v.c. cilindrica}
    \textit{taxa de condução dentro do v.c.}=q^{\prime \prime}_z r \delta \phi \delta r
\end{equation}

\begin{equation}\label{eq: taxa de conducao fora v.c. cilindrica}
    \textit{taxa de condução para fora do v.c.}= q^{\prime \prime}_{r + \delta r} r \delta \phi \delta r
\end{equation}
Fazendo a mesma expansão, temos
\begin{equation}
    q^{\prime \prime}_{z+\delta z}=q^{\prime \prime}|_z + \frac{\partial}{\partial}[q^{\prime \prime}|_z \delta z]
\end{equation}

Fazendo \eqref{eq: taxa de conducao dentro v.c. cilindrica} - \eqref{eq: taxa de conducao fora v.c. cilindrica}

\begin{align}
    \eqref{eq: taxa de conducao dentro v.c. cilindrica}_z - \eqref{eq: taxa de conducao fora v.c. cilindrica}_{z+ \delta z}  &= q^{\prime \prime}_z r \delta \phi \delta r - \left[ q^{\prime \prime}_z + \frac{\partial}{\partial z}q^{\prime \prime}_z \delta z \right] r \delta \phi \delta r\\
    &= - \frac{\partial}{\partial z}q^{\prime \prime}_z \delta z r \delta \phi \delta r
\end{align}

Similarmente para \(r\)

\begin{equation}
    q^{\prime \prime}_{r+\delta r}=q^{\prime \prime}|_r + \frac{\partial}{\partial}[q^{\prime \prime}_r|_r \delta r]
\end{equation}

\begin{align}
    \eqref{eq: taxa de conducao dentro v.c. cilindrica}_r - \eqref{eq: taxa de conducao fora v.c. cilindrica}_{r+ \delta r} &= q^{\prime \prime}_r r \delta \phi \delta r - \left[ q^{\prime \prime}_r + \frac{\partial}{\partial r}q^{\prime \prime}_r \delta r \right] r \delta \phi \delta r\\
    &= - \frac{\partial}{\partial r}\left( r q^{\prime \prime}_r \right)  \delta r \delta \phi \delta z
\end{align}

E para \(\phi \) 

\begin{equation}
    q^{\prime \prime}_{\phi+\delta \phi}=q^{\prime \prime}|_\phi + \frac{\partial}{\partial}[q^{\prime \prime}|_\phi  r \delta \phi]
\end{equation}

\begin{align}
    \eqref{eq: taxa de conducao dentro v.c. cilindrica}_{\phi }  - \eqref{eq: taxa de conducao fora v.c. cilindrica}_{\phi + \delta \phi }  &= q^{\prime \prime}_\phi r \delta \phi \delta r - \left[ q^{\prime \prime}_\phi + \frac{\partial}{\partial}q^{\prime \prime}_\phi r \delta \phi \right] r \delta \phi \delta r\\
    &= - \frac{\partial}{\partial \phi }q^{\prime \prime}_\phi  \delta \phi \delta r  \delta z
\end{align}

Logo, substituindo nossa equação na forma mais geral e isolando para o armazenamento, vamos ter
\begin{equation}
    \eqref{eq: calor armazenado v.c. cilindrica} = \eqref{eq: calor gerado v.c. cilindrica} + \eqref{eq: taxa de conducao dentro v.c. cilindrica} - \eqref{eq: taxa de conducao fora v.c. cilindrica}
\end{equation}
\begin{flalign}
        \frac{\partial}{\partial t}(\rho c T) r \delta \phi \delta r \delta z = \dot{q}_{GER}r \delta \phi \delta r \delta z - \left( \frac{\partial}{\partial z}q^{\prime \prime}_z \delta z r \delta \phi \delta r + \frac{\partial}{\partial r}\left( r q^{\prime \prime}_r \right)  \delta r \delta \phi \delta z
        + \frac{\partial}{\partial \phi }q^{\prime \prime}_\phi r  \delta \phi \delta r  \delta z \right)  
\end{flalign}
Fazendo o mesmo desenvolvimento chegamos em

\begin{equation}
    \frac{1}{\alpha }  \frac{\partial T}{\partial t} = \frac{\dot{q}_{GER} }{k} + \left[ \frac{\partial ^{2}  T }{\partial z ^{2} } + \frac{1}{r} \frac{\partial }{\partial r} \left( r \frac{\partial T}{\partial r}  \right) + \frac{1}{r^{2} } \frac{\partial ^{2}  T }{\partial \phi  ^{2} } \right] 
\end{equation}
Fazendo as mesmas considerações que foram feitas para o caso retangular, vamos considerar regime
estacionário, sem geração de calor e um caso com condução unidimensional, apenas na direção radial.
Assim, temos
\begin{equation}\label{eq: conducao unidemensional cilindrica}
    \frac{1}{r} \frac{\partial }{\partial r}  \left(r \partial\frac{\partial T}{\partial r} \right) = 0
\end{equation}
Vamos resolver essa equação por meio das nossas condições de contorno.
\begin{equation}\label{eq: condicoes de contorno cilindrica}
    \begin{cases}
        T(r = r_0) = T_{0} \\
        T(r= r_i) = T_{i}
    \end{cases}
\end{equation}
Resolvendo nossa equação diferencial para o caso unidimensional, temos
\begin{align}
    \frac{1}{r} \frac{\partial }{\partial r}  \left(r \partial\frac{\partial T}{\partial r} \right) &= 0\\
    \frac{\partial }{\partial r}  \left(r \partial\frac{\partial T}{\partial r} \right) &= 0\\
    \int\frac{\partial}{\partial r}   \left(r \partial\frac{\partial T}{\partial r} \right) &= 0\\
    r \frac{\partial T}{\partial r} &= C_1\\
    \frac{\partial T}{\partial r} &= \frac{C_1}{r}\\
    \partial T &= \frac{C_1}{r} \partial r\\
    \int \partial T &= \int \frac{C_1}{r} \partial r\\
    T(r) &= C_1 \ln r + C_2
\end{align}
Aplicando \eqref{eq: condicoes de contorno cilindrica} temos
\begin{align}
    T_i &= C_1 \ln r_i + C_2\\
    C_2 &= T_i - C_1 \ln r_i\\
    T_0 &= C_1 \ln r_0 + C_2\\
    &\therefore\\
    T_0 &= C_i \ln r_0 + T_i - C_1 \ln r_i\\
    T_0 - T_i &= C_1 \ln \frac{r_0}{r_i}\\
    C_1 &= \frac{T_0 - T_i}{\ln \frac{r_0}{r_i}}
\end{align}
Portanto, substituindo na nossa solução temos
\begin{align}
    T(r) &= \frac{T_0 - T_i}{\ln \frac{r_0}{r_i}} \ln r + T_i - \frac{T_0 - T_i}{\ln \frac{r_0}{r_i}} \ln r_i
    T - T_i &= \frac{T_0 - T_i}{\ln \frac{r_0}{r_i}} \ln \frac{r}{r_i}
\end{align}
Relembrando da nossa equação de Fourier para o caso cilíndrico, temos
\begin{equation}
    q_r = - k A \frac{\partial T}{\partial r}
\end{equation}
Onde \(A = 2 \pi r L\). Substituindo nossa solução para \(T(r)\) na equação de Fourier, temos
\begin{equation}
    q_r = - 2 k \pi L \left[\frac{T_i - T_0}{\ln \left(\frac{r_0}{r_i}\right)}\right]
\end{equation}
Para nossa associação em serie e em paralelo de paredes, nosso raciocínio se da o mesmo, então nao
me destrincharei novamente. Lembrando que a resistência térmica tem formula
\begin{equation}
    R_T = \frac{\ln \frac{r_o}{r_i}  }{2\pi k l} 
\end{equation}
\subsection{Condução em um corpo esférico}
Como o raciocínio se da extremamente similar aos outros casos, nao farei de novo, apenas dando as
formulas ja deduzidas e fica a critério do leitor caso ele queira conferir. Nossa equação geral da
condução em paredes esférica se da como
\begin{equation}\label{eq: equacao geral da conducao em paredes esfericas}
    \frac{1}{\alpha } \frac{\partial T}{\partial t} = \frac{\dot{q}_{GER} }{k} + \left[ \frac{1}{r^{2} } \frac{\partial}{\partial r} \left( r^2 \frac{\partial T}{\partial r} \right) + \frac{1}{r^{2} \sin \theta } \frac{\partial}{\partial \theta} \left( \sin \theta \frac{\partial T}{\partial \theta} \right) + \frac{1}{r^{2} \sin ^{2} \theta } \frac{\partial ^{2} T}{\partial \phi ^{2}} \right]
\end{equation}
Para o caso mais simples, onde o regime se da como permanente, sem geração de calor e com condução
unidimensional, vamos ter a seguinte equação
\begin{equation}
    \frac{1}{r^2} \partial\frac{\partial }{\partial r} \left(r^2 partia\frac{\partial T}{\partial r} \right) = 0 
\end{equation}
Com as condições de contorno
\begin{equation}
    \begin{cases}
        T(r = r_e) = T_{e} \\
        T(r= r_i) = T_{i}
    \end{cases}
\end{equation}
Logo, nossa equação de condução se da
\begin{align}
    q_r = \frac{T_i - T_e}{\left(\frac{\left(\frac{1}{r_i} - \frac{1}{r_e}  \right)}{4 \pi k}\right)} 
\end{align}
Similarmente com associação em serie e em paralelo, o raciocínio se da o mesmo, então nao farei o
desenvolvimento. Lembrando apenas que a resistência térmica tem formula
\begin{equation}
    R_T = \frac{\left(\frac{1}{r_i} - \frac{1}{r_e}\right)}{4\pi k} 
\end{equation}
\section{O conceito de resistência térmica condutiva e convectiva}
Suponha que tenhamos uma placa que esteja entre duas temperaturas diferentes. A temperatura maior
tem valor \(T_{S1}\)  e a de menor valor \(T_{S2}\). A placa tem uma espessura \(L\). Ao longo da
placa, o calor flui por meio da condução e o calor que chega a placa vem, majoritariamente, por meio
da convecção. A resistência térmica da placa, \(R_{T. Cond}\), pode ser calculada da lei de Fourier
tendo formula
\begin{equation}\label{eq: resistencia termica condutiva placa}
    R_{T. Cond} = \frac{L}{kA}
\end{equation}
A resistência térmica convectiva, \(R_{T. Conv}\), pode ser calculada da lei de newton tendo formula
\begin{equation}\label{eq: resistencia termica convectiva placa}
    R_{T. Conv} = \frac{1}{hA}
\end{equation}
A taxa de transferência total de calor, \(q_{x}\), pode ser dada pela consideração de cada elemento,
ou seja, a taxa condutiva, a taxa convectiva do ambiente de maior temperatura e a taxa convectiva do
ambiente de menor temperatura. A taxa de transferência total de calor, \(q_{x}\), pode ser dada por
\begin{equation}\label{eq: taxa de transferencia total de calor}
    q = \frac{T_{\infty_1} - T_{\infty_2}}{R_{T. Conv_1} + R_{T. Cond} + R_{T. Conv_2}}
\end{equation}
Onde \(T_{\infty_1 }, T_{\infty_2}\) sao respectivamente as temperaturas do ambiente de maior e
menor. Substituindo \eqref{eq: resistencia termica condutiva placa} e \eqref{eq: resistencia termica
convectiva placa} na \eqref{eq: taxa de transferencia total de calor} temos
\begin{equation}\label{eq: taxa de transferencia total de calor 2}
        q = \frac{T_{\infty_1} - T_{\infty_2}}{\frac{1}{h_1A} + \frac{L}{kA} + \frac{1}{h_2A}} = \frac{T_{\infty_1} - T_{\infty_2}}{\frac{1}{h_1} + \frac{L}{k} + \frac{1}{h_2}}
\end{equation}
Onde \(h_1,h_2\) sao as taxas de convecção do ambiente de maior e menor temperatura. Similarmente,
para um tubo cilíndrico, a resistência térmica condutiva, \(R_{TCond}\), se da como
\begin{equation}\label{eq: resistencia termica condutiva tubo}
    R_{TCond} = \frac{\ln \left(\frac{r_0}{r_1} \right)}{2\pi k L}
\end{equation}
E suas resistências convectivas se dao de forma igual. Porem agora serão chamadas de \(T_1, T_2\)
onde \(T_1\) se da como o mais quente e \(T_2\) como o mais frio. Nossa equação de \(q_x\) se da
\begin{align}\label{eq: taxa de transferencia total de calor cilindrica nao global}
    q &= \frac{T_1 - T_2}{R_{TConv_1} + R_{TCond} + R_{TConv_2}}\\
    &= \frac{T_1 - T_2}{\left(\frac{1}{h_i A_i}\right) + \left(\frac{\ln \left(\frac{r_0}{r_1} \right)}{2\pi k L}\right) + \left(\frac{1}{h_0 A_0}\right)}
\end{align}
Ou de maneira global
\begin{equation}\label{eq: taxa de transferencia total de calor global}
    q = U_i A_i \left(T_1 - T_2\right) = \frac{\left(T_1 - T_2\right)}{\frac{1}{U_i A_i}} 
\end{equation}
Onde \(U_i\) se da como o coeficiente global de transferência de calor. Obtido quando igualamos
\eqref{eq: taxa de transferencia total de calor cilindrica nao global} e \eqref{eq: taxa de transferencia total de calor global}
\begin{equation}
    \frac{1}{U_i} = \left(\frac{1}{h_i}\right) + \left(\frac{r_i \ln \left(\frac{r_0}{r_1} \right)}{k}\right) + \left(\frac{r_i}{h_0r_0}\right)
\end{equation}
O coeficiente global também pode estar baseado na área externa do tubo, dos dando a equação
\begin{equation}\label{eq: equacao global para tubo area externa}
    q = U_0 A_0 \left(T_1 - T_2\right) = \frac{\left(T_1 - T_2\right)}{\frac{1}{U_0 A_0}}
\end{equation}
Onde igualando \eqref{eq: equacao global para tubo area externa} e \eqref{eq: taxa de transferencia
total de calor global} vamos obter
\begin{equation}
    \frac{1}{U_0 A_0} = \left(\frac{1}{h_i}\right) + \left(\frac{\ln \left( \frac{r_0}{r_i} \right) }{2 \pi K L}\right) + \left(\frac{1}{h_0 A_0}\right)
\end{equation}
Onde 
\begin{equation}
    \begin{dcases}
        A_i = 2 \pi r_i L ;\\
        A_0 = 2\pi r_0 L  ;\\
    \end{dcases}
\end{equation}
Portanto
\begin{equation}
    \frac{1}{U_0} = \frac{r_0}{h_{i} r_{i} } + \frac{r_0 \ln \left( \frac{r_0}{r_i} \right)}{K} + \frac{1}{h_0}
\end{equation}
\subsection{Aletamento como forma de Aumentar eficiência}
Da nossa equação \eqref{eq: taxa de transferencia total de calor cilindrica nao global} temos
algumas formas de aumentar a eficiência do sistema, dentre elas
\begin{itemize}
    \item Reduzindo \(RT_{conv_i} \) 
    \item Reduzindo \(RT_{cond} \)
    \item Reduzindo \(RT_{conv_0} \)
\end{itemize}
Diminuir \(RT_{conv_i} \) diversas vezes não é muito viável, por mais que seja possível. Por sua
vez, reduzir \(RT_{cond} \) envolve mudar a espessura da parede e possivelmente o material, o que
também não é muito viável. Porém, se fossemos pegar \(RT_{cond_0} \) é mais possível, aumentando-se
a área de contato externa da troca térmica. Isso pode ser feito através de aletas. Aletas são
expansões, pequenas, nas paredes dos trocadores de calor. Vamos partir diretamente para o balanço de
energia em uma aleta. \par

Suponha uma aleta instalada numa superfície de temperatura \(T_s\) sendo ela resfriada ao longo de
seu comprimento por uma temperatura \(T_\infty \). Pegando um elemento de volume dessa aleta e
considerando que
\begin{itemize}
    \item Regime Permanente
    \item Troca de calor Unidirecional
    \item Ao longo do material temos uma condutividade térmica constante
    \item Radiação desprezível
    \item Efeitos de geração de calor ausentes
    \item Coeficiente convectivo uniforme
\end{itemize}
No nosso elemento de volume, podemos escrever o balanço de energia como
\begin{equation}
    q_{cond_x} = q_{cond_{x + dx}} + q_{conv_x}
\end{equation}
Usando o conceito de derivada parcial, vamos ter que
\begin{equation}
    q_{cond_{x+dx} } = q_{cond_x} + \frac{\mathrm{d}q_{cond_x} }{\mathrm{d}x} \mathrm{d} x
\end{equation} 
Substituindo então, vamos ter
\begin{align}
    q_{cond_x} &= q_{cond_x} + \frac{\mathrm{d}q_{cond_x} }{\mathrm{d}x} \mathrm{d} x + q_{conv_x}\\
    - \frac{\mathrm{d}q_{cond_x} }{\mathrm{d}x} \mathrm{d} x &= q_{conv_x}\\
\end{align}
Substituindo a Lei de Fourier e a Lei de Resfriamento de Newton, vamos ter
\begin{align}
    \frac{\mathrm{d}}{\mathrm{d}x} \left( -k A_{tr} \frac{\mathrm{d}T}{\mathrm{d}x}  \right) \mathrm{d} x &= h A_{tr} \left(T_s - T_\infty \right)\\
    k A_{tr} \frac{\mathrm{d}^2 T}{\mathrm{d}x^2} + k \frac{\mathrm{d}T}{\mathrm{d}x} \frac{\mathrm{d}A_{tr} }{\mathrm{d}x} &= h \frac{\mathrm{d}A_s}{\mathrm{d}x} \left(T_s - T_\infty \right)\\
\end{align}
Onde \(A_{tr} \) é a área transversal ao fluxo e \(\mathrm{d} A_s\) é a area superficial do elemento
diferencial. Dividindo nossos termos por \(k A_{tr} \)
\begin{equation}
    \frac{\mathrm{d}^2 T}{\mathrm{d}x^2} + \frac{1}{A_{tr} } \frac{\mathrm{d}T}{\mathrm{d}x} \frac{\mathrm{d}A_{tr} }{\mathrm{d}x} = \frac{h}{k A_{tr}} \frac{\mathrm{d}A_s}{\mathrm{d}x} \left(T_s - T_\infty \right)
\end{equation}
Para resolvermos essa equação, vamos precisar impor algumas condições sobre a geometria da aleta.
Para aletas com seção transversal constante, como pinos, vamos ter que
\begin{align}
    \frac{\mathrm{d}^{2} T}{\mathrm{d}x^{2} } = \frac{h}{k A_{tr}} \frac{\mathrm{d}A_s}{\mathrm{d}x} \left(T_s - T_\infty \right)\\
\end{align}
Onde \(\frac{\mathrm{d}A_{s} }{\mathrm{d}x}  = P\), logo
\begin{equation}
    \frac{\mathrm{d}^{2} T}{\mathrm{d}x^{2} } - \frac{h P}{k A_{tr}} \left(T_s - T_\infty \right) = 0
\end{equation}
Para resolver essa equação, vamos adimensionaliza-la, transformando a variável dependente, ou seja,
colocando uma temperatura de excesso
\begin{equation}
    \theta (x) = \left( T(x) - T_\infty  \right) \Rightarrow  T(x) = \theta (x) + T_\infty
\end{equation}
Substituindo na equação, vamos ter
\begin{align}
    \frac{\mathrm{d}}{\mathrm{d}x} \left( \frac{\mathrm{d}T}{\mathrm{d}x}  \right) - \frac{hP}{kA_{tr}}\left( T - T_\infty  \right) &= 0
    \frac{\mathrm{d}}{\mathrm{d}x} \left( \frac{\mathrm{d}}{\mathrm{d}x} \left( \theta + T_\infty  \right)  \right) - \frac{hP}{kA_{tr}}\theta &= 0\\
    \frac{\mathrm{d}\theta }{\mathrm{d}x}  - \frac{hP}{kA_{tr}}\theta &= 0\\
\end{align}
Agrupando os termos constantes, vamos ter 
\begin{equation}
    m^{2} = \frac{hP}{kA_{tr}} \Rightarrow m = \sqrt{(hP)/(kA_{tr})} 
\end{equation}
Onde esse é chamado o coeficiente da aleta. Substituindo na equação, vamos ter
\begin{equation}
    \frac{\mathrm{d}\theta }{\mathrm{d}x}  - m^{2}\theta = 0
\end{equation}
A solução dessa equação diferencial nos da
\begin{equation}
    \theta (x) = C_1 e^{mx} + C_2 e^{-mx}
\end{equation}
Nossas condições de contorno então são
\begin{align}
\begin{dcases}
    x = 0, & \theta = \theta_s ;\\
    x \to \infty , & \theta = 0 ;\\
\end{dcases}
\end{align}
Onde nossa segunda condição de contorno pode se dar para 4 casos. O primeiro a ser tratado é esse,
de aleta muito longa. Logo, podemos achar nossas constantes. Para a nossa primeira condição de contorno
\begin{align}
    \theta_s &= C_1 e^{m0}  + C_2 e^{-m0} \\
    C_1 &= \theta_s - C_2
\end{align}
Substituindo na segunda condição de contorno, vamos ter
\begin{align}
    0 &= C_1 e^{m\infty} + C_2 e^{-m\infty}\\
    C_1 &= 0\\
    C_2 &= \theta_s
    & \therefore\\
    \theta (x) &= \theta_s e^{-mx}  
\end{align}
A partir dessa equação, se quisermos a taxa de transferência de calor através da aleta, vamos
considerar que o calor conduzido na raiz da aleta é igual ao da convecção na superfície da haste
para o fluido. Logo, vamos ter
\begin{align}
    q_{aleta} &= - k A_{tr} \frac{\mathrm{d}\theta}{\mathrm{d}x}|_{x = 0}\\
    &= \int_{0}^{\infty} h P \mathrm{d}x \theta\\
    &\therefore\\
    q_{aleta} &= \sqrt{h p A_{tr} k} \theta_s
\end{align}
Para o nosso segundo caso, vamos considerar a extremidade da aleta como isolada. Nossas condições de
contorno ficam, então
\begin{align}
    \begin{dcases}
        x = 0 , & \theta = \theta_s ;\\
        x = L , & \frac{\mathrm{d}\theta}{\mathrm{d}x} = 0 ;\\
    \end{dcases}
\end{align}
Substituindo na nossas equações, vamos ter
\begin{align}
    \frac{\mathrm{d}\theta}{\mathrm{d}x} &= C_1 e^{mx} - C_2 e^{-mx}\\
    0 &= mC_1 e^{mL} + mC_2 e^{-mL}\\
\end{align}
Onde \(C_1 = \frac{\theta_s}{1 + e^{2ml}} \) e \(C_2 = \frac{\theta_s}{1 + e^{ -}_2ml}  \), portanto
\begin{equation}
    \theta (x) = \theta_s \left( \frac{e^{mx}}{1 + e^{2ml}} + \frac{e^{-mx}}{1 + e^{- 2ml} }\right) \Rightarrow \theta = \theta_s \frac{\cosh \left[m \left(L - x\right)\right]}{\cosh \left(mL\right)} 
\end{equation}
E nossa taxa se da como
\begin{equation}
    q_{aleta} = \sqrt{h p A_{tr} k} \theta_s \frac{\tanh \left(mL\right)}{mL}
\end{equation}
Para a nossa proxima condição, vamos considerar que tenha convecção na extremidade da aleta, logo,
nossas condições ficam
\begin{equation}
    \begin{dcases}
        x = 0 ,& \theta = \theta_s ;\\
        x = L ,& - k A_{tr} \frac{\mathrm{d}\theta}{\mathrm{d}x}|_{x = L} = h \theta\
    \end{dcases}
\end{equation}
Nossas duas equações ficam
\begin{align}
\theta &= \theta_s \left[\frac{\cosh \left[m \left(L - x\right) + \left(k/mk\right) \sinh \left[m\left(L - x\right)\right]\right]}{\cosh \left(mL\right) + \left(k/mk\right)\sinh \left(mL\right)}  \right]\\
q_{aleta} &= \sqrt{h p A_{tr} k} \theta_s \frac{\sinh \left(mL\right) + \left(h/mk\right)\cosh \left(mL\right)}{\cosh \left(mL\right) + \left(h/mk\right)\sinh \left(mL\right)}
\end{align}
Para o nosso ultimo caso, nossa temperatura na extremidade da aleta é especificada, logo, nossas
condições de contorno ficam
\begin{align}
    \begin{dcases}
        x = 0 ,& \theta = \theta_s ;\\
        x = L ,& \theta = \left(T_L - T_\infty \right)
    \end{dcases}
\end{align}
Nossas duas equações ficam
\begin{align}
    \theta &= \theta_s \left[\frac{\frac{\theta_L}{\theta_S}\sinh(mx) + \sinh \left[m\left(L - x\right)\right]}{\sinh \left(mL\right)} \right]\\
    q_{aleta} &= \sqrt{h P k A_{tr}} \theta_s \left[\frac{\cosh \left(mL\right) - \left(\frac{\theta_L}{\theta_S}\right)}{\sinh\left(mL\right)}\right]
\end{align}
\subsubsection{m das Aletas}
Para uma aleta retangular m vale. \(P\)  é o perímetro, \(A_{tr} \) é a área de seção transversal e
\(e\) é a espessura.
\begin{align}
    P &= 2b + 2e \approx 2b\\
    A_{tr} &= be\\
    m &= \sqrt{\frac{2h}{ke}}
\end{align}
Para uma aleta curva, \(r\) é o raio da aleta e \(e\) é a espessura
\begin{align}
    P &= 2(2\pi r) + 2e \approx 4\pi r\\
    A_{tr} &= 2\pi r e\\
    m &= \sqrt{\frac{2h}{ke}} \left(1 + \frac{1}{2} \frac{e}{H} \right)
\end{align}
Para uma de pino
\begin{align}
    P &= 2\pi r\\
    A_{tr} &= \pi r^2\\
    m &= \sqrt{\frac{2h}{kr}}
\end{align}
\subsubsection{Desempenho Térmico de uma Aleta}
A efetividade da aleta é definida como a razão entre a taxa de transferência de calor na aleta e a
taxa de transferência de calor se a aleta nao estivesse la. Logo, vamos ter
\begin{equation}
    \varepsilon = \frac{q_{aleta}}{h A_{trb} \theta_s}
\end{equation}
Onde \(A_{trb}\) é a area da base da aleta. Para uma aleta infinita, vamos ter
\begin{equation}
    \varepsilon = \left(\frac{kP}{h A_tr}\right)^{\frac{1}{2}}
\end{equation}
Para a eficiência da aleta, vamos ter
\begin{equation}
    \eta = \frac{q_{aleta}}{h A_{aleta}\theta_s}
\end{equation}
Para uma aleta infinita, vamos ter
\begin{equation}
    \eta = \frac{\sqrt{hPkA_tr}\theta_S}{hPL\theta_s} \Rightarrow \eta = \sqrt{\frac{kA_{tr}}{hP}} \frac{1}{L}
\end{equation}
Para uma aleta de pino, vamos ter
\begin{equation}
    \eta = \frac{\tanh\left(mL\right)}{mL} 
\end{equation}
\subsubsection{Transferência de Calor em uma superfície aletada}
Para uma superfície aletada, vamos ter
\begin{equation}
    q = q_R + q_{aleta}
\end{equation}
Vamos ter, no fim
\begin{equation}
    q = h \left(A_R + \eta A_{aleta}\right)\left(T_S - T_\infty\right)
\end{equation}
Onde \(A_R\) é a area da superfície sem a aleta, ou seja, vai ser a area da superfície menos a area
da aleta.
\section{Convecção Forcada}
Relembrando nossa formula de convecção, vamos ter que \(q = hA(T_s - T_\infty)\), onde \(A\) é a
area, \(T_s\) é a temperatura da superfície e \(T_\infty\) é a temperatura do fluido. O ponto
importante é o coeficiente \(h\) que é o coeficiente de transferência de calor por convecção. Até o
momento, tivemos valores dados, mas agora, vamos conseguir calcula-lo.\par

Porem, esse calculo é difícil, pois depende de muitos fatores, como o movimento do fluido. Temos 2
tipos de movimentos, os naturais e os forcados. 
\begin{itemize}
    \item {Os naturais se da com a convecção livre, como brisas e a densidade.}
    \item {Os forcados se da com a convecção forçada, como ventiladores e bombas.}
\end{itemize}
Podemos relacionar alguns números  adimensionais com esse processo, como o numero de Reynolds,
Prandtl e Nusselt. O numero de Reynolds é dado por
\begin{equation}
    R_e = \frac{\rho \upsilon d_c}{\mu}
\end{equation}
Para o escoamento no interior de um tubo, vamos ter
\begin{equation}
    R_e = \frac{4m}{\pi \mu D }
\end{equation}
Temos então a relação que
\begin{align}
    \begin{dcases}
        R_e < 2100 ,& \text{escoamento laminar}\\
        R_e > 4000 ,& \text{escoamento turbulento}\\
        2100 \leq  R_e \leq  4000 ,& \text{transição}
    \end{dcases} 
\end{align}
Para um escoamento sobre uma placa plana, vamos ter
\begin{align}
    \begin{dcases}
        R_e \leq  500000 ,& \text{escoamento laminar}\\
        R_e > 500000 ,& \text{escoamento turbulento}
    \end{dcases}
\end{align}
O numero de Nusselt, por sua vez correlaciona as duas formas de transferência de calor. Por exemplo,
\begin{align}
    \begin{dcases}
        q_{cond} = KA \frac{\Delta T}{l}
        q_{conv} = hA\Delta T
    \end{dcases} 
\end{align}
Dividindo as duas, vamos ter
\begin{equation}
    \frac{q_{conv}}{q_{cond}} = \frac{hA\Delta T}{KA \frac{\Delta T}{l}} = \frac{hl}{K} = N_u
\end{equation}
Portanto, o numero de Nusselt é a razão entre a convecção e a condução. Dado pela formula
\begin{equation}
    N_u = \frac{hd_c}{k}
\end{equation}
Onde \(d_c\) é o diâmetro característico. O numero de Prandtl  correlaciona as espessuras das
camadas limite térmica e hidrodinâmica. É dado por
\begin{equation}
    P_r = \frac{\mu C}{k}
\end{equation}
\(C\) é o calor especifico do fluido \(K\) é a condutividade térmica do fluido.Para \(P_r = 1\) a
camada limite térmica e hidrodinâmica tem a mesma espessura. Para \(P_r \ll 1\) a difusividade
térmica \( \gg \) a difusividade de momentum (quantidade de movimento). 
\subsection{Relação entre os adimensionais}
A relação empírica entre os tres adimensionais se da
\begin{equation}
    N_u = C R_e^m P_r^n
\end{equation}
Onde \(C, m\) e \(n\) sao função da geometria da superfície e do tipo de escoamento. Para um
escoamento laminar, na região de entrada ou escoamento completamente desenvolvido, vamos ter
\begin{equation}
    N_{Nu} = 1.86 \left(N_{Re} N_{Pr} \frac{d_c}{L} \right)^{0.33} \left(\frac{\mu_b}{\mu_w} \right)^{0.14}
\end{equation}
Com \(\mu_b\) sendo a viscosidade do fluido com temperatura igual na parede e \(\mu_w\) sendo a
viscosidade do fluido com temperatura media do fluido. 
Para escoamento transiente, vamos ter
\begin{equation}
    N_{Nu} = \frac{(\frac{f}{8})(N_{Re} - 1000)N_{pr}}{1 + 12.7(\frac{f}{8})^{1/2}(N_{pr}^{2/3} - 1)}
\end{equation}
Com \(f\) tendo equação:
\begin{equation}
    f = \frac{1}{(0.79 \ln(N_{Re}) - 1.64)^2}
\end{equation}
Para escoamento turbulento, vamos ter
\begin{equation}
    N_{Nu} = 0.023 N_{Re}^{0.8} N_{Pr}^{0.33}\left(\frac{\mu_b}{\mu_w} \right)
\end{equation}
Para o entorno de uma esfera, vamos ter
\begin{equation}
    N_{Nu} = 2 + 0.6 N_{Re}^{1/2} N_{Pr}^{0.33}
\end{equation}
\subsection{Metodologia para o calculo de h}
Para o calculo de \(h\) é boa pratica seguirmos os passos
\begin{enumerate}
    \item {Identificar a geometria da superfície solida em contanto com o fluido e suas dimensões características}
    \item {identificar o fluido e sua temperatura media do escoamento}
    \item {Obter as propriedades físicas e térmicas do fluido na temperatura media de escoamento}
    \item {Calcular o Numero de Reynolds}
    \item {Selecionar a correlação empírica apropriada, de acordo com as informações disponíveis}
\end{enumerate}

\subsubsection{Exemplo}
Água escoa a \(0.02 kg \cdot s\), sendo aquecida de 20 a 60 \(\degree C\). O tubo tem \(2.5 cm\) de
diâmetro e \(1 m\) de comprimento. A parede do tubo esta a \(90\degree C\). \par

A geometria é cilíndrica e a \(d_c\) é o diâmetro. O fluido escoando é a aguá, com \(T_m = 40\degree
C\). As propriedades físicas e térmicas da aguá a \(40\degree C\) sao
\begin{align}
    \begin{dcases}
        \mu_{40} = 0.625 \cdot 10^{-3} kg \cdot \frac{m}{s}\\
        k = 0.540 W/m \cdot K\\
        \rho = 992.2 \frac{kg}{m^3}\\
        P_r = 4.34\\
        \mu_{90} = 0.315 \cdot 10^{-3} kg \cdot \frac{m}{s}
    \end{dcases}
\end{align}
O numero de Reynolds é
\begin{equation}
    N_{Re} = \frac{4m}{\pi \mu D} = \frac{4 \cdot 0.02}{\pi \cdot 0.625 \cdot 10^{-3} \cdot 0.025} = 1629.74
\end{equation}
Nosso escoamento é laminar, portanto vamos usar a correlação
\begin{equation}
    N_{Nu} = 1.86 \left(N_{Re} N_{Pr} \frac{d_c}{L} \right)^{0.33} \left(\frac{\mu_b}{\mu_w} \right)^{0.14}
\end{equation}
Portanto, temos
\begin{equation}
    N_{Nu} = 1.86 \left(1629.74 \cdot 4.34 \cdot \frac{0.025}{1} \right)^{0.33} \left(\frac{0.625 \cdot 10^{-3}}{0.315 \cdot 10^{-3}} \right)^{0.14} = 11.29
\end{equation}
Voltando na nossa relação, vamos ter então
\begin{equation}
    N_{Nu} = \frac{h d_c}{k} \Rightarrow h = \frac{N_{Nu} k}{d_c} = \frac{11.29 \cdot 0.627}{0.025} = 283 \frac{W}{m^2 \cdot K}
\end{equation}
\section{Determinação de h para processo de convecção livre}
Para o caso de convecção livre tal fenômeno ocorre devido ao gradiente de densidade do fluido. Ou
seja, por meio da variação do \(\rho \) ao longo do fluido, variação essa causada pela diferença de
temperatura quando o fluido esta em contato com a superfície solida. \par

Com o aumento da temperatura, a densidade do fluido diminui, fazendo com que apareçam forcas
flutuantes que causam o movimento do fluido. Algumas equações empíricas uteis para predizer o h sao:
\begin{equation}
    N_{u} = \frac{H d_{c} }{k} = a (Ra)^n \begin{dcases}
    Ra, &\quad Gr \cdot Pr ;\\
    Gr, &\quad Gr = \frac{(d_{c} )^{3} \rho ^{2} g \beta \Delta T}{\eta ^{2} } .\\
    \end{dcases}
\end{equation}
Onde o \(\beta \) é o coeficiente de expansão térmica, \(\Delta T\) é a diferença de temperatura
entre a superfície solida e o fluido, \(d_c\) é a dimensão característica, \(g\) é a aceleração da
gravidade, \(\eta \) é a viscosidade dinâmica, \(\rho \) é a densidade e \(k\) é a condutividade
térmica e \(Ra\) é o numero de Grashof multiplicado pelo numero de Prandtl. \par

\subsection{Correlações do Numero de Nusselt para convecção natural}
\subsubsection{Placa vertical}
Dimensão característica: \(L\)
\begin{align}
    10^4 < Ra < 10^9 &\Rightarrow N_{Nu} = 0.59 Ra^{1/4}\\
    10^9 < Ra < 10^{13} &\Rightarrow N_{Nu} = 0.1 Ra^{1/3}
\end{align}
\subsubsection{Placa Inclinada}
Dimensão característica: \(L\)
Igualmente ao anterior, apenas multiplicamos por \(g\) e \(\cos \alpha \) para \(Ra < 10^9\)
\begin{equation}
    Gr = \frac{g \beta \Delta T L^3 \cos \alpha \rho ^{2} }{\eta ^{2} }
\end{equation} 
\subsubsection{Placa Horizontal com superficie quente superior}
Dimensão característica: \(\frac{A}{P}\)
\begin{align}
    10^4 < Ra < 10^7 &\Rightarrow N_{Nu} = 0.54 Ra^{1/4}\\
    10^7 < Ra < 10^{11} &\Rightarrow N_{Nu} = 0.15 Ra^{1/3}
\end{align}
\subsubsection{Placa Horizontal com superficie quente inferior}
Dimensão característica: \(\frac{A}{P}\)
\begin{align}
    10^5 < Ra < 10^{11 } &\Rightarrow N_{Nu} = 0.27 Ra^{1/4}\\
\end{align}
\subsubsection{Cilindro vertical}
Dimensão característica: \(L\)
Pode ser tratado como placa vertical se
\begin{equation}
    D \geq \frac{35L}{Gr^{1/4} }    
\end{equation}
\subsubsection{Cilindro horizontal}
Dimensão característica: \(D\)
\begin{align}
    10^{-5} < Ra < 10^{12} &\Rightarrow N_{Nu} =  0.6 + \frac{0.387 Ra^{1/6} }{\left[ 1 + \left( \frac{0.559}{Pr} \right)^{\frac{9}{16}}  \right]^{\frac{8}{27}} }
\end{align}
\subsubsection{Esfera}
Dimensão característica: \(\frac{1}{2}\pi  D\)
\begin{align}
    Ra \leq 10^{11} &\Rightarrow N_{Nu} = 2 + \frac{0.589 Ra^{1/4} }{\left[ 1 + \left( \frac{0.469}{}Pr \right)^{9/16}  \right]^{4/9} }\\
\end{align}
\subsubsection{Exemplo}
Estimar o coeficiente de convecção para um cilindro horizontal de 0.1 m de diâmetro. Sendo a
temperatura de superfície do cilindro de \(130 \degree C\) e a temperatura do ar ambiente de
\(30\degree C\).   
\subsubsection{Solucao}
A dimensão característica sera \(d_{c} = 0.1\). A temperatura media do ar sera \(T_{m} = 80\degree C\).
Vamos obter as propriedades físicas e térmicas do fluido na temperatura media de escoamento.
\begin{align}
    \rho _{m} &= 1 \frac{kg}{m^{3} }\\
    \eta _{m} &= 20.934 \cdot 10^{-6}  \frac{kg}{m \cdot s}\\ 
    P_r &= 0.708\\
    k_{m} &= 0.02984 \frac{W}{m \cdot \degree C}\\
    \beta _{m} &= 0.00283 \frac{1}{\degree C}
\end{align}
Vamos calcular \(Gr\) e \(Ra\):
\begin{align}
    Gr &= ((d_{c} )^{3} \rho ^{2} g \beta \Delta T)/(\eta ^{2} )\\
    &= ((0.1)^{3} (1)^{2} (9.81) (0.00283) (130 - 30))/((20.934 \cdot 10^{-6} )^{2} )\\
    &= 6335063.95\\
    Ra &= Gr \cdot Pr\\
    &= 6335063.95 \cdot 0.708 = 4485225.2766\\
\end{align}
Vamos calcular \(N_{u} \)
\begin{align}
    N_{u} &= 0.6 + \frac{0.387 Ra^{1/4} }{\left[ 1 + \left( \frac{0.559}{Pr} \right)^{9/16}  \right]^{\frac{4}{9}} }\\
    &= 0.6 + \frac{0.387 (4485225.2766)^{1/6} }{\left[ 1 + \left( \frac{0.559}{0.708} \right)^{9/16}  \right]^{\frac{8}{27}} } = 4.72492\\
\end{align}
Vamos calcular o coeficiente de convecção:
\begin{align}
    h &= \frac{N_{u} k}{d_{c} }\\
    &= \frac{4.72492 (0.02984)}{0.1} = 1.404 \frac{W}{m^{2} \cdot \degree C}\\
\end{align}
\subsubsection{Exemplo 2}
Calcule a taxa de perda de calor por convecção das partes superior e inferior de uma placa plana e
horizontal de 1 \(m^{2} \) de area aquecida a \(220 \degree C\) em um ambiente a \(20 \degree C\).
\subsubsection{Solução}
A dimensão característica sera \(d_{c} = \frac{A}{P} = 1/4 = 0.25\). A temperatura media do ar sera \(T_{m} = 120\degree C\).
Vamos obter as propriedades físicas e terminal do fluido na temperatura media de escoamento.
\begin{align}
    \rho _{m} &= 0.898 \frac{kg}{m^{3} }\\
    \eta _{m} &= 22.671 \cdot 10^{-6}  \frac{kg}{m \cdot s}\\
    P_r &= 0.7\\
    K_{m} &= 0.03274 \frac{W}{m \cdot \degree C}\\
    \beta _{m} &= 0.00255 \frac{1}{\degree C}
\end{align}
Vamos calcular \(Gr\) e \(Ra\):
\begin{align}
    Gr &= \frac{(d_{c} )^{3} \rho ^{2} g \beta \Delta T}{\eta ^{2}}\\
    &= \frac{(0.25)^{3} (0.898)^{2} (9.81) (0.00255) (220 - 20)}{(22.671 \cdot 10^{-6} )^{2}} = 122650840.577339\\
    &= 122650840.578\\
    Ra &= Gr \cdot Pr\\
    &= 122650840.578 \cdot 0.7 = 85855688.4048 \leq 10^7\\
\end{align}
Vamos calcular \(N_{u} \)
\begin{align}
    N_{u} &= 0.54 (Ra)^{1/4}\\
    &= 0.54 (85855688.404137)^{1/4} = 51.9799704801454
\end{align}
Vamos calcular o coeficiente de convecção:
\begin{align}
    h &= \frac{N_{u} k}{d_{c} }\\
    &= \frac{51.98 (0.03274)}{0.25} = 6.802 \frac{W}{m^{2} \cdot \degree C}\\
\end{align}
\section{Análise Transiente}
O estudo de transferência de calor transiente é importante pois nos leva a entender a transferência
e a variação da temperatura ao longo do tempo. Esse efeito acontece quando as condições de contorno
se alteram com o tempo. \par
A primeira coisa que veremos sera o \emph{Adimensional de Biot}. Esse adimensional e importante pois
vai nos dizer qual taxa de transferência sera maior, ou seja, se a condução ou convecção
prevalecera. Se definirmos \(T_i\) como a temperatura interna e \(T_{a} \) como a temperatura
ambiente. Se tivermos \(T_{i} < T_{a} \) o adimensional sera definido como
\begin{align}
    B_{i} = \frac{\frac{L_{s} }{K}}{\frac{1}{h} } = \frac{h L_{s} }{K},
    \begin{dcases}
    B_{i} > 40 &\Rightarrow \; \frac{1}{h} \text{ é desprezível }  ;\\
    0.1 < B_{i} < 40 &\Rightarrow \;  \text{ Resistências Finitas }  ;\\
    B_{i} < 0.1, &\Rightarrow \; \frac{d_{c} }{k} \text{Desprezível } .
    \end{dcases}
\end{align}
Em que \(L_{s} \) é volume por area e \(h\) é a condutividade térmica do solido.  Onde \(L_{s} \)  é
definido como 
\begin{equation}
    L_{s} = \frac{V}{A}
\end{equation}
Onde \(V\) é o volume e \(A\) é a area. \par
\subsection{Método da capacitância Global}
\emph{So funciona se \(B_{i} < 0.1\) }
Se tivermos um resfriamento de um corpo que começa em \(t=0\) a temperatura diminui com \(t > 0\)
ate atingir o valor \(T_{\infty} \) e essa redução deve-se ao coeficiente de transferência de calor
por convecção na interface solido-liquido. \par

A hipótese do método da capacitância global e que a temperatura do corpo seja espacialmente uniforme
em qualquer instante de tempo do processo transiente, admitindo que os gradientes de temperatura no
interior do solido seja desprezível. \par

A taxa de transferência de calor por convecção na interface solido-liquido é dada por
\begin{equation}
    \dot{Q} = \overline{h} A (T_{s} - T_{\infty} )
\end{equation}
A taxa de variação de energia interna do corpo é dada por
\begin{equation}
    \dot{Q}  \rho C V \frac{dT(t)}{dt}
\end{equation}
Igualando as duas equações acima temos
\begin{equation}
    \overline{h} A (T_{s} - T_{\infty} ) = \rho C V \frac{dT(t)}{dt}
\end{equation}
Introduzindo a temperatura adimensional
\begin{equation}
    \theta = \left( T(t) - T_\infty  \right) 
\end{equation}
Introduzindo a temperatura adimensional na equação temos
\begin{equation}
    \frac{\mathrm{d}\theta (t)}{\mathrm{d}t} = -\frac{\overline{h} A}{\rho  C V}\theta (t) 
\end{equation}
Fazendo a separação de variáveis e integrando temos
\begin{align}
    \int_{\theta _{0} }^{\theta} \frac{1}{\theta} \mathrm{d}\theta &= -\int_{0}^{t} \frac{\overline{h} A}{\rho  C V} \mathrm{d}t\\
    \ln{(\theta)} - \ln{(\theta_{0})} &= -\frac{\overline{h} A}{\rho  C V} t\\
    \ln{\left( \frac{\theta}{\theta_{0}} \right) } &= -\frac{\overline{h} A}{\rho  C V} t\\
    \frac{\theta}{\theta_{0}} &= e^{-\frac{\overline{h} A}{\rho  C V} t}\\
\end{align}
Vamos definir que
\begin{align}
    \frac{\overline{h} At}{\rho  C V} &= \frac{\overline{h} t}{\rho C L} \cdot \left( \frac{k}{k} \right) \cdot \left( \frac{L}{L} \right) \\ 
    &= \frac{\overline{h} L \alpha t}{kL^{2} } = B_{i} Fo
\end{align}
Em que \(Fo\) é o numero de Fourier Substituindo na nossa equação, vamos ter
\begin{equation}\label{eq:capacitancia global}
    \theta = \theta_0 e^{-B_{i} Fo}
\end{equation}
\subsubsection{Exemplo}
Uma placa de alumínio de \(K = 160 \frac{W}{m \degree C}, \rho = 2790 \frac{kg}{m^{3} } , C = 0.88
\frac{kJ}{kg \degree C} \) com \(3 cm\) de espessura e temperatura uniforme \(T_0 = 225 \degree C\) é
atingindo por um fluido de \(T_\infty = 25 \degree C\) com \(h = 320 \frac{W}{m^2 \degree C}\).
Determine \(t\) para que a temperatura da placa seja \(T = 50 \degree C\). \par
\subsubsection{Solução}
Primeiro vamos calcular o \(L_{s} \) 
\begin{align}
    L_{s} &= \frac{V}{2A_{s} }\\
    &= \frac{A_{s} e}{2A_{s} }\\
    &= \frac{e}{2} = 1.5 \cdot 10^{-2} m
\end{align}
Agora vamos calcular a difusividade térmica
\begin{align}
    \alpha &= \frac{k}{\rho  C}\\
    &= \frac{160}{2790 \cdot 880} = 6.1 \cdot 10^{-5} \frac{m^{2} }{s}
\end{align}
Agora vamos calcular o numero de Biot
\begin{align}
    Bi &= \frac{h L_{s} }{k}\\
    &= \frac{320 \cdot 1.5 \cdot 10^{-2} }{160} = 0.03
\end{align}
Podemos usar o método da capacitância global, pois \(Bi < 0.1\). \par
Vamos calcular nosso \(\theta \) 
\begin{align}
    \theta &= 50 - 25 = 25 \degree C\\
    \theta_{0} &= 225 - 25 = 200 \degree C
    &\therefore\\
    \theta &= 200 e^{-0.03 Fo}
    \frac{25}{200} &= e^{-0.03 \cdot Fo} \\
    Fo = \frac{-2.08}{-0.03} = 69.633
\end{align}
Agora vamos calcular o tempo
\begin{align}
    Fo &= \frac{\alpha t}{L^{2} } \\
    t &= \frac{Fo L^{2} }{\alpha} \\
    &= \frac{69.633 \cdot 0.015^{2} }{6.1 \cdot 10^{-5} } = 256.8430 \; s = 4.28 \; min 
\end{align}
\subsubsection{Exemplo 2}
Uma esfera de \(1 \; mm\) de diâmetro, condutiva térmica de \(k = 25 \frac{W}{m \degree C}\) massa
especifica de \(\rho  = 8400 \frac{kg}{m^{3} }\) e calor especifico de \(C = 0.4 \frac{kJ}{kg \degree
C}\) \(h = 560 \frac{W}{m^{2} \degree C}\) e \(\frac{\theta}{\theta_0} = 0.01\). Qual o tempo que
isso leva. Qual a temperatura do centro geométrico apos \(10 \; s\) com \(T_0 = 80\degree\) e
\(T_\infty = 25\degree\)  ? \par
\subsubsection{Solução}
Calculando \(\alpha \) 
\begin{align}
    \alpha &= \frac{k}{\rho  C}\\
    &= \frac{25}{8400 \cdot 0.4} = 7.44 \cdot 10^{-6} \frac{m^{2} }{s}
\end{align}  
Calculando \(L_{s} \)
\begin{align}
    L_{s} &= \frac{V}{A_{s} } \\
    &= \frac{V}{4 \pi r^{2} } \\
    &= \frac{\frac{4}{3} \pi r^{3} }{4 \pi r^{2} } \\
    &= \frac{r}{3} = \frac{1}{3} \cdot \frac{10^{-3}}{2} = 1.67 \cdot 10^{-4} m 
\end{align}
Calculando \(Bi\)
\begin{align}
    Bi &= \frac{h L_{s} }{k}\\
    &= \frac{560 \cdot 1.67 \cdot 10^{-4} }{25} = 0.0037408
\end{align}
Achando o \(Fo\)
\begin{align}
    \frac{\theta}{\theta_{0}} &= e^{-B_{i} Fo}\\
    \ln{\left( \frac{\theta}{\theta_{0}} \right) } &= -B_{i} Fo\\
    Fo &= \frac{-\ln{\left( \frac{\theta}{\theta_{0}} \right) }}{B_{i}} \\
    &= \frac{-\ln{\left( 0.01 \right) }}{0.0037408} = 1231.066
\end{align}
Calculando o tempo
\begin{align}
    Fo &= \frac{\alpha t}{L^{2} } \\
    t &= \frac{Fo L^{2} }{\alpha} \\
    &= \frac{1231.066 \cdot (1.67 \cdot 10^{-4})^{2}  }{7.44 \cdot 10^{-6} } = 4.6146 \; s
\end{align}
Para sabermos a temperatura apos \(10 \; s\) vamos achar \(Fo\) novamente
\begin{align}
    Fo &= \frac{\alpha t}{L^{2} } \\
    Fo &= \frac{7.44 \cdot 10^{-6} \cdot 10}{(1.67 \cdot 10^{-4})^{2} } = 2667.71845
\end{align}
Agora vamos achar \(\theta \) 
\begin{align}
    \frac{\theta}{\theta_{0}} &= e^{-B_{i} Fo}\\
    \theta &= \theta_{0} e^{-B_{i} Fo}\\
    &= 80 e^{-0.0037408 \cdot 2667.71845} = 0.0037
\end{align}
Substituindo na equação de temperatura
\begin{align}
    T &= \theta + T_{\infty} \\
    &= 0.0037 + 25 = 25.0037 \degree C
\end{align}  
\section{Condição de contorno mista}
Podemos estudar essa aplicação quando uma parte da fronteira esta sujeita a uma condição de
convecção e a outra um fluxo de calor. \par

Consideramos uma placa de espessura L, inicialmente a temperatura \(T_0\) e para qualquer instante
\(t>0\) , recebe um calor em uma das suas superficies a um fluxo constante \(q^{\prime \prime}\) e
ao mesmo tempo dissipe calor por convecção na outra superfície a uma temperatura \(T_{\infty}\) e
um coeficiente de convecção \(h\). O balanco de energia pode ser dado por:
\begin{align}
    A \cdot q^{\prime \prime} + \overline{h} A (T_\infty - T(t)) &= \rho C A L \frac{\mathrm{d}T(t)}{\mathrm{d}t}\\
    q^{\prime \prime} + \overline{h} (T_\infty - T(t)) &= \rho C L \frac{\mathrm{d}T(t)}{\mathrm{d}t}\\ 
\end{align}
Adimensionalizando, temos
\begin{equation}
    \theta (t) = T(t) - T_{\infty}
\end{equation}
Rearranjando a equação, temos
\begin{align}
    \frac{\mathrm{d}\theta (t)}{\mathrm{d}t} + m \theta (t) &= Q
     \begin{dcases}
     m, &= \frac{h}{\rho C} ;\\
     Q, &=\frac{q^{\prime \prime}}{\rho C L} .\\
     \end{dcases}
\end{align}
A solução da equação diferencial é dada por
\begin{equation}
    \theta (t) = \theta_0 e^{-mt} + 1 \left( 1 - e^{-mt} \right) \frac{q^{\prime\prime} }{h} 
\end{equation}
\subsubsection{Exemplo 3}
Um ferro tem propriedades dadas por
\begin{itemize}
    \item \(C = 450 \; \frac{J}{Kg \degree C}\)
    \item \(K = 70 \; \frac{W}{m \degree}\) 
    \item \(\rho = 7840 \; \frac{kg}{m^{3} }\) 
    \item \(m = 1 \; kg\)
    \item \(A =\; m ^{2} \) 
    \item \(h = 50 \; \frac{W}{m^{2} \degree C}\)
    \item \(T_\infty = 20 \degree C\) 
    \item \(T_0 = 20 \degree C\)
    \item \(q = 250 \; W\)
    \item \(t = 300.0 \; s\)    
\end{itemize}
Qual a temperatura do ferro? Qual a temperatura de equilíbrio do ferro? 
\subsubsection{Solução}
Calculando \(L\)
\begin{align}
    L &= \frac{V}{A}\\
    L &= \frac{m}{A\rho} = \frac{1}{A \rho} = \frac{1}{0.025 \cdot 7840}\\
    &= 0.0051 \; m
\end{align}
Calculando \(Bi\)
\begin{align}
    Bi &= \frac{h L}{K}\\
    &= \frac{50 \cdot 0.0051}{70}  = 0.003643
\end{align}  
\(Bi \; < \; 0.1\). Vamos calcular \(m\) 
\begin{align}
    m &= \frac{h}{\rho C}\\
    &= \frac{50}{7840 \cdot 450 \cdot 0.0051} = 0.00278
\end{align}
Calculando \(q^{\prime\prime}\)
\begin{align}
    q^{\prime\prime} &=  q \cdot \frac{1}{A}\\
    &= \frac{250}{0.025} = 10000
\end{align}
Achando \(\theta _0\) 
\begin{align}
    \theta _0 &= T_0 - T_\infty \\
    &= 20 - 20 = 0
\end{align}
Calculando \(\theta (t)\)
\begin{align}
    \theta (t) &= \theta_0 e^{-mt} + 1 \left( 1 - e^{-mt} \right) \frac{q^{\prime\prime} }{h} \\
    &= 0 + 1 \left( 1 - e^{-0.00278 \cdot 300} \right) \frac{10000}{50} = 113.14 \; \degree C \\
\end{align}
Calculando \(T\) 
\begin{align}
    T &= \theta + T_\infty \\
    &= 113.14 + 20 = 133.14 \; \degree C
\end{align}
Nossa temperatura de equilíbrio é dada por
\begin{align}
    \theta (\infty ) &= \lim_{t \to \infty}  \theta_0 e^{-mt} + 1 \left( 1 - e^{-mt} \right) \frac{q^{\prime\prime} }{h} \\
    \theta _\infty &= \frac{q^{\prime\prime} }{h} \\
    &= \frac{10000}{50} = 200 \; \degree C
\end{align}
Por fim, \(T\)
\begin{align}
    T &= \theta _\infty + T_\infty \\
    &= 200 + 20 = 220 \; \degree C
\end{align} 
\subsection{Uso de cartas de Temperaturas Transientes}
Em muitas situações, os gradientes de temperaturas no interior dos sólidos nao pode ser desprezado e
nao pode ser aplicado uma analise global do sistema ( \(Bi > 0.1\) ). Nesse caso, a analise dos
problema envolver descobrir o gradiente de temperatura dentro do solido, fazendo o problema ficar
extremamente mais complexo. \par

Para resolver esse problema, podemos usar o método dos ábacos, a partir dos adimensionais \(Bi\),
\(Fo\) e temperatura adimensional \(\frac{\theta}{\theta _0}\). \par

%adicionar os gráficos aqui

\subsubsection{Exemplo 4}
Uma placa de ferro com \(5 \; cm\) com \(C= 460\; \frac{J}{kg\degree C}, \; K = 60 \; \frac{W}{m
\degree C}, \rho = 7850 \; \frac{kg}{m^3}\)  esta a \(T_{i} = 225 \degree C\) e \(T_{\infty} = 25 \degree C\), com \(h = 500 \; \frac{W}{m^{2} \: \degree C}\)  
Calcule a temperatura do centro em \(t = 120 \; s\), a temperatura a \(1 \; cm\) no mesmo tempo e a
energia removida por metro quadrado da placa. \par
\subsubsection{Solução}
Calculando \(\alpha \) 
\begin{align}
    \alpha &= \frac{K}{\rho C}\\
    &= \frac{60}{7850 \cdot 460}  = 1.662 \cdot 10^{-5}
\end{align}
Calculando \(L\)
\begin{align}
    L &= \frac{V}{A}\\
    &= \frac{A_{s} e}{2A_{s} }\\
    &= \frac{e}{2} = \frac{0.05}{2} = 0.025 \; m
\end{align}
Calculando \(Bi\)
\begin{align}
    Bi &= \frac{h L}{K}\\
    &= \frac{500 \cdot 0.025}{60}  = 0.208333
\end{align}
\(Bi > 0.1\) temos que utilizar o ábaco. Vamos calcular o \(Fo\)
\begin{align}
    Fo &= \frac{\alpha t}{L^{2} }\\
    &= \frac{1.662 \cdot 10^{-5} \cdot 120}{0.025^{2} } = 3.19104
\end{align}
Para o ábaco, vamos precisar do inverso de \(Bi\)
\begin{align}
    \frac{1}{Bi} &= \frac{K}{h L}\\
    &= \frac{60}{500 \cdot 0.025} = 4.8
\end{align}
Pelo ábaco, vamos achar o \(\frac{\theta}{\theta _0}\) para \(Fo = 3.19104\) e \(\frac{1}{Bi} =
4.8\) que vai ser \(0.55\). \par
Vamos ter então que
\begin{align}
    \theta_0 = T_i - T_\infty = 225 - 25 = 200 \; \degree C
\end{align}
Portanto temos que
\begin{align}
    \frac{\theta}{\theta _0} &= 0.55\\
    \theta &= 0.55 \cdot 200 = 110 \; \degree C
\end{align}
Calculando \(T\)
\begin{align}
    T &= \theta + T_\infty \\
    &= 110 + 25 = 135 \; \degree C
\end{align}
Vamos achar a temperatura a \(1 \; cm\) da placa. Para isso, vamos usar o outro ábaco, onde vamos
precisar do inverso de \(Bi\) que vale \(4.81\) e vamos precisar de \(\frac{x}{L} =
\frac{0.015}{0.025} = 0.6\) onde \(x\) é a distancia do centro da placa. \par Pelo ábaco, vamos
achar o \(\frac{\theta}{\theta _0}\) para \(\frac{1}{Bi}, \frac{x}{L}\), com valor de \(0.95\) \par
Vamos ter que
\begin{align}
    \frac{\theta}{\theta _0} &= 0.95 \cdot 0.55\\
    = 0.5225
\end{align}
Pois vai ser \(0.95\) em relação a distância do centro da placa. Podemos achar \(\theta \)
\begin{align}
    \theta &= 0.5225 \cdot 200 = 104.5 \; \degree C
\end{align}
E agora \(T\)
\begin{align}
    T &= \theta + T_\infty \\
    &= 104.5 + 25 = 129.5 \; \degree C
\end{align}
Vamos achar a energia removida por metro quadrado da placa. Para isso, vamos usar o outro ábaco,
onde vamos precisar de \(Bi, \;Bi^{2} Fo\), portanto vamos precisar de \(Bi^{2} Fo = 0.208333^{2}
\cdot 3.19104  = 0.1385\). Olhando pelo ábaco, vamos achar o \(\frac{Q}{Q_0} = 0.4\). Portanto, por
meio da equação \(Q = mc \Delta T\), conseguimos substituir na nossa equação, nos dando
\begin{align}
    Q_0 &= m C \left( T_0 - T_\infty  \right)\\
    &= \rho V C \left( T_0 - T_\infty  \right)\\
    &= \rho A_s e C \left( T_0 - T_\infty  \right)\\
    &= 7850 \cdot 0.05 \cdot 460 \left( 225 - 25  \right) = 36110000\\
\end{align}
Portanto, temos que
\begin{align}
    Q &= 0.4 \cdot 36110000 = 14444000 \; J
\end{align}
\subsubsection{Exemplo 5}
Uma esfera de \(5 \; cm\) de diâmetro com \(C= 460\; \frac{J}{kg\degree C}, \; K = 60 \; \frac{W}{m}
\degree C, \rho = 7850 \; \frac{kg}{m^3}\)  esta a \(T_{i} = 225 \degree C\) e \(T_{\infty} = 25
\degree C\), com \(h = 1000 \; \frac{W}{m^{2} \: \degree C}\) \par
Calcule a temperatura do centro em \(t = 120 \; s\), a temperatura a \(1 \; cm\) do centro no mesmo
tempo.
\subsubsection{Solução}
Primeiros vamos calcular \(\alpha \)
\begin{align}
    \alpha &= \frac{K}{\rho C}\\
    &= \frac{60}{7850 \cdot 460} = 1.662 \cdot 10^{-5}
\end{align}
Vamos calcular \(L\)
\begin{align}
    L &= \frac{V}{A}\\
    &= \frac{\frac{4}{3} \pi r ^{3}  }{4 \pi r^{2} }
    &= \frac{r}{3} = \frac{0.025}{3} = 0.00833 
\end{align}
Vamos calcular \(Bi\)
\begin{align}
    Bi &= \frac{h L}{K}\\
    &= \frac{1000 \cdot 0.00833}{60} = 0.1388
\end{align}
\(Bi > 0.1\) temos que utilizar o ábaco. Por ser geometria esférica, vamos ignorar o que fizemos e
vamos recalcular \(Bi\) com a dimensão sendo o \(r\). Portanto
\begin{align}
    Bi &= \frac{h r}{K}\\
    &= \frac{1000 \cdot 0.025}{60} = 0.4167
\end{align}
\(Bi > 0.1\) temos que utilizar o ábaco. Vamos calcular o inverso de \(Bi\)
\begin{align}
    \frac{1}{Bi} &= \frac{K}{h r}\\
    &= \frac{60}{1000 \cdot 0.025} = 2.4
\end{align}
Vamos calcular \(Fo\)
\begin{align}
    Fo &= \frac{\alpha t}{r^{2} }\\
    &= \frac{1.662 \cdot 10^{-5} \cdot 120}{0.025^{2} } = 3.19104
\end{align}
Pelo ábaco da esfera, vamos achar o \(\frac{\theta}{\theta _0}\) para \(Fo = 3.19104\) e
\(\frac{1}{Bi} = 2.4\), vamos ter um valor de \(0.026\)  
\subsubsection{Exemplo 6 }
Qual o tempo necessário em segundos para que uma placa cilíndrica de diâmetro \(D = 0.05 m\)
inicialmente a \(20 \degree C\) seja aquecida a \(300 \degree C\) se a temperatura do ar ambiente
vale \(T_{\infty} =800\). \(h = 85 \frac{W}{m^{2} \degree C}\,, k= 60.37 \; W \;, \alpha = 1.75 \cdot 10^{-5} \frac{m^2}{s} \)
\subsubsection{Solução}
Primeiro vamos calcular \(L\)
\begin{align}
    L &= \frac{D}{4}\\
    &= \frac{0.05}{4} = 0.0125  
\end{align}
Vamos calcular \(Bi\)
\begin{align}
    Bi &= \frac{h L}{k}\\
    &= \frac{85 \cdot 0.0125}{60.37} = 0.0175998012257744
\end{align}
Temos que \(Bi < 0.1\), portanto vamos utilizar a equação
\begin{align}
    \theta &= \frac{T - T_\infty}{T_0 - T_\infty}\\
    &= \frac{300 - 20}{800 - 20}  = 0.358974358974359
\end{align}
Vamos achar \(Fo\)
\begin{align}
    \theta_0 = e^{-Bi Fo} \implies Fo = \frac{\ln \left( \frac{\theta_0}{\theta} \right)}{-Bi}\\
    &= \frac{\ln  (0.356) }{-0.0175998012257744} = 25.26
\end{align}
Pela formula \(Fo = \frac{\alpha t}{L^{2} }\), vamos achar \(t\)
\begin{align}
    t &= \frac{Fo L^{2} }{\alpha}\\
    &= \frac{25.26 \cdot 0.0125^{2} }{1.75 \cdot 10^{-5} } = 225.535714285714
\end{align}
\section{Projeto de Trocador de Calor}
Um trocador de calor (TC) é um dispositivo para implementar a troca de calor entre 2 fluidos, que estão
em diferentes temperaturas. Os projetos de TC dependem da analise térmica, projetos mecânico
preliminar e o projeto de fabricação. \par

\subsection{Media Logarítmica da Diferença de Temperatura (MLDT)}
A MLDT é uma media ponderada das diferenças de temperatura entre os fluidos. A MLDT é definida como
\begin{equation}\label{eq: MLDT}
    MLDT = \frac{\Delta T_{\max } - \Delta T_{\min }}{\ln \left( \frac{\Delta T_{\max }}{\Delta T_{\min }} \right)} 
\end{equation}
Para correntes paralelas e correntes opostas, a temperatura assume os seguintes comportamentos
% Colocar o Gráfico
\subsubsection{Exemplo}
Um fluido quente entra num TC a \(900 \degree C\) e sai a \(600 \degree C\). Um fluido frio entra a
\(100 \degree C\) e sai a \(500 \degree C\) qual o MLDT para correntes paralelas e oposta,
respectivamente?
\subsubsection{Solucao}
Para correntes paralelas o \(\Delta T_{\max }\) vale
\begin{align}
    \Delta T_{\max } &= T_{h,i} - T_{c,i}\\
    &= 900 - 100 = 800
\end{align}
E \(\Delta T_{\min }\) vale
\begin{align}
    \Delta T_{\min } &= T_{h,o} - T_{c,o}\\
    &= 600 - 500 = 100
\end{align}
Por fim MLDT vale
\begin{align}
    MLDT &= \frac{\Delta T_{\max } - \Delta T_{\min }}{\ln \left( \frac{\Delta T_{\max }}{\Delta T_{\min }} \right)} \\
    &= \frac{800 - 100}{\ln \left( \frac{800}{100} \right)} = 336.6288
\end{align}
Para correntes opostas o \(\Delta T_{\min }\) vale
\begin{align}
    \Delta T_{\min } &= T_{h,i} - T_{c,o}\\
    &= 900 - 500 = 400
\end{align}
E \(\Delta T_{\max }\) vale
\begin{align}
    \Delta T_{\max } &= T_{h,o} - T_{c,i}\\
    &= 600 - 100 = 500
\end{align}
Portanto MLDT vale
\begin{align}
    MLDT &= \frac{\Delta T_{\max } - \Delta T_{\min }}{\ln \left( \frac{\Delta T_{\max }}{\Delta T_{\min }} \right)} \\
    &= \frac{500 - 400}{\ln \left( \frac{500}{400} \right)} = 448.1420
\end{align}
\subsection{Balanço de Energia}
Supondo que oTC esteja bem isolado, o balanco de energia da taxa de calor cedido para a taxa de
calor recebido vale
\begin{equation}
    -\dot{q}_{ced} = \dot{q}_{rec}
\end{equation}
Onde
\begin{equation}
    \dot{q} = \dot{m} c (t_{s} - t_{e} )
\end{equation}
Logo
\begin{equation}
    \dot{m}c(t_{e} - t_{s} ) = \dot{M}C(T_{s} - T_{e} )
\end{equation}
O calor trocado entre os fluidos através das superficies dos tubos pode ser obtido através das
resistências térmicas 
\begin{equation}
    \dot{q} = \frac{T_{s} - T_{e} }{R_{t} } = \frac{(\Delta T)_{total}}{\frac{1}{h_{i} A_{i} } + R_{cond} + \frac{1}{h_{e} A_{e}  }}
\end{equation}
Ou ainda
\begin{equation}
    \dot{q} = U_{0c}A_{e} (\Delta T)_{total} \Rightarrow \dot{q} = U_{0c}A_{i} MLDT
\end{equation}
Onde \(U_{0c}\) é o coeficiente global de transferência de calor, \(A_{e}\) é a area de troca de
calor e \(MLDT\) é a media logarítmica da diferença de temperatura.
\subsubsection{Exemplo}
Um fluido líquido \(C_{p} = 4 \frac{kj}{Kg \degree C}\) escoa a 0.5 \(Kg/s\) em um TC, entrando a
\(20 \degree C\) e saindo a \(60 \degree C\). No espaço anular, escoa água quente a 1
\(\frac{Kg}{s}\) a qual entra a \(90 \degree C\). Sabendo que as correntes sao paralelas e que o
calor especifico da água é \(4.18 \frac{kJ}{kg \degree C}\), determine a temperatura de saída da
água, MLDT, comprimento do trocador de calor, sabendo que \(U_{0C} = 2000 \frac{w}{m ^{2} \degree C}
\) e o diâmetro externo do tubo é \(0.05 m\).
\subsubsection{Solução}
Primeiro vamos achar a temperatura de saída da água
\begin{align}
    -\dot{q}_{\acute{a}gua} &= \dot{q}_{l\acute{i}quido}\\
    \dot{m}_{\acute{a}gua} c_{\acute{a}gua} (t_{\acute{a}gua,e} - t_{\acute{a}gua,s} ) &= \dot{m}_{l\acute{i}quido} c_{l\acute{i}quido} (t_{l\acute{i}quido,s} - t_{l\acute{i}quido,e} )\\
    1 \cdot 4180 \cdot (90 - t_{\acute{a}gua,s} ) &= 0.5 \cdot 4000 \cdot (60 - 20) \\
    t_{\acute{a}gua, s}= 70.86124 \; \degree C
\end{align} 
Portando, podemos achar o MLDT
\begin{align}
    MLDT &= \frac{\Delta T_{max} - \Delta T_{min}}{\ln \left( \frac{\Delta T_{max}}{\Delta T_{min}} \right)} \\
    &= \frac{70 - 10.86124}{\ln \left( \frac{70}{10.86124} \right)} = 31.7388 \; \degree C
\end{align}
Agora podemos achar o comprimento do TC. Primeiro vamos achar o calor trocado
\begin{align}
    \dot{q} &= \dot{M} C (T_{s} - T_{e} )\\
    &= 0.5 \cdot 4000 (60 - 20) = 80000 \; W
\end{align}
Achando o comprimento agora
\begin{align}
    \dot{q} &= U_{0c} A_{i} MLDT\\
    \dot{q} &= U_{0c} \pi D L MLDT\\
    L &= \frac{\dot{q}}{U_{0c} \pi D MLDT}\\
    &= \frac{80000}{2000 \pi 0.05 \cdot 31.7388} = 8.02323 \; m
\end{align}
%Fazer para o outro caso
\subsection{Fator de Fuligem (Incrustação)}
O fator de fuligem é um fator que leva em conta a incrustação que ocorre nas superficies dos tubos
devido a deposição de fuligem. A incrustação reduz a condutividade térmica do material e aumenta a
resistência térmica. A resistência pode ser calculada por
\begin{equation}
    \dot{q} = \frac{A_{e} (\Delta T)_{total}}{\frac{1}{h_{i} } + \frac{1}{h_{e} } + R_{d} }
\end{equation}
Onde \(R_{d}\) é a soma do fator de fuligem interno e externo. O coeficiente global de transferência
de calor, levando em conta a fuligem é dado por
\begin{align}
    U_{0d} &= \frac{1}{\frac{1}{h_{i} } + \frac{1}{h_{e} } + R_{d} }\\
    \frac{1}{U_{0d} } &= \frac{1}{U_{0c} } + R_{d} = \frac{1}{U_{0c} } + R_{di} + R_{de}
\end{align}
Onde \(R_{di}\) e \(R_{de}\) sao os fatores de fuligem interno e externo, respectivamente. Para o
calculo de transferência de calor, vamos usar o coeficiente global sujo, sendo assim temos
\begin{equation}
    \dot{q} = U_{0d} A_{e} (MLDT)
\end{equation}
\subsubsection{Exemplo}
Um Tc vai aquecer 9820 \(\frac{Kg}{s}\) de benzeno (\(c = 0.425 \frac{J}{kg \degree  C}\) ) de \(80 \degree c\)
para \( 120 \degree C \), utilizando o tolueno (\(c = 0.44 \frac{J}{kg\degree c}\) ), o qual é
resfriado de \(160 \degree  C\) para \(100 \degree C\). O fator de fuligem é
\(0.001 \left( \frac{W}{m^{2} \degree c} \right)^{-1}  \) e deve ser considerado para cada fluxo. O 
coeficiente limpo é \(U_{0c} = 149 \frac{W}{m^{2} \degree c} \). Se tem trocadores bi-tubulares de 
20 metros de comprimento com tubos de area \(0.435 m^{2} \). Qual a vazão de tolueno necessária e 
quantos trocadores sao necessários?
\subsubsubsection{Solucao}
Primeiro vamos achar a masa de tolueno necessário
\begin{align}
    \dot{M}C(T_{s} - T_{e} ) &= \dot{m}c(t_{e} - t_{s} )
\end{align}   
   


